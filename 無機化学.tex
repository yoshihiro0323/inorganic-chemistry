\documentclass[dvipdfmx,a4paper,twocolumn]{jsarticle}
\usepackage[top=20truemm,bottom=20truemm,left=20truemm,right=20truemm]{geometry}
\usepackage{multirow}
\usepackage[version=3]{mhchem}
\usepackage{color}
\usepackage{enumerate}
\usepackage{tikz}
\usepackage{lscape}
\usepackage{amsmath}
\usepackage{otf}

\newcommand{\hl}[1]{\underline{\textcolor{red}{\gtfamily #1}}}
\newcommand{\shl}[1]{{\textcolor{blue}{#1}}}
\pagestyle{empty}

\newcommand{\K}{%
  \begin{tikzpicture}[baseline=(A.base),font=\sffamily\small]
    \node[draw=blue, rectangle, rounded corners=2pt,
          text=blue,
          fill=blue!10!white,
          line width=.4pt, inner sep=1pt, outer sep=0pt
          ] (A) {工業的製法};
  \end{tikzpicture}%
}
\newcommand{\R}{%
  \begin{tikzpicture}[baseline=(A.base),font=\sffamily\small]
    \node[draw=red, rectangle, rounded corners=2pt,
          text=red,
          fill=red!10!white,
          line width=.4pt, inner sep=1pt, outer sep=0pt
          ] (A) {例};
  \end{tikzpicture}%
}

\columnseprule=0.3mm

\begin{document}

\twocolumn[
\begin{center}
\Huge{\gtfamily 無機化学}\\
\vspace{5mm}
\end{center}
\part{非金属元素}
]
 \section{水素}
  \hl{無色無臭}の\hl{気体}\footnote{融点 14K 沸点 20K} \hl{最も軽く}、水に溶け\hl{にくい}
  \subsection{同位体}
  \ce{_{}^{1}H} 99\%以上 \ce{_{}^{2}H} (\hl{D})0.015\% \ce{_{}^{3}H} (\hl{T})微量
  \subsection{製法}
  \begin{itemize}
   \item ナフサの電気分解 \K
  \item 赤熱した\hl{コークス}に\hl{水蒸気}を吹き付ける \K\\
  \shl{\ce{C + H2O -> H2 + CO}}
  \item \hl{水}(\hl{水酸化ナトリウム水溶液})の電気分解\\
  \shl{\ce{2H2O -> 2H2 + O2}}
  \item \hl{イオン化傾向}が\hl{\ce{H2}より大きい}金属と希薄強酸\\
  \R \;\ce{Fe + 2HCl -> FeCl2 + H2 ^}\\
  \R \;\ce{Zn + 2HCl -> ZnCl2 + H2 ^}
 \end{itemize}
 \subsection{反応}
 \begin{itemize}
  \item 水素と酸素(爆鳴気の燃焼)\\
  \shl{\ce{2H2 + O2 -> H2O}}
  \item 加熱した酸化銅(\ajRoman{2})と水素\\
  \shl{\ce{CuO + H2 -> Cu + H2O}}
  \item 水酸化ナトリウムと水\\
  \shl{\ce{NaH + H2O -> NaOH + H2}}
 \end{itemize}
 \newpage
 \section{貴ガス}
 \hl{\ce{He}}, \hl{\ce{Ne}}, \hl{\ce{Ar}}, \hl{\ce{Kr}}, \ce{Xe}, \ce{Rn}
  \subsection{性質}
  \begin{itemize}
   \item 無色・無臭
   \item 第18族元素であり、電子配置がオクテットを満たすため反応性が低い。
   \item イオン化エネルギーが極めて大きい。
   \item 電子親和力は\hl{極めて小さい}(\hl{ほぼ0})。
   \item 電気陰性度は\hl{定義されない}。
  \end{itemize}
  \subsection{生成}
  \ce{_{}^{40}K}の電子捕獲\\
  \quad\shl{\ce{_{}^{40}K + e- -> _{}^{40}Ar}}
  \subsection{ヘリウム \ce{He}}
  浮揚ガス
  \subsection{ネオン \ce{Ne}}
  ネオンサイン
  \subsection{アルゴン \ce{Ar}}
  \ce{N2}, \ce{O2}に次いで3番目に空気中での存在量が多い(約1\%)。
 \newpage
 \section{ハロゲン}
  \subsection{性質}
  \begin{center}
  \rotatebox{90}{
  \begin{tabular}{|c||c|c|c|c|}\hline
   単体の化学式&\ce{F2}&\ce{Cl2}&\ce{Br2}&\ce{I2}\\ \hline
   分子量&\multicolumn{1}{c}{小}&\multicolumn{2}{c}{\ce{<->}}&\multicolumn{1}{c|}{大}\\ \hline
   分子間力(反応性)&\multicolumn{1}{c}{弱(強)}&\multicolumn{2}{c}{\ce{<->}}&\multicolumn{1}{c|}{強(弱)}\\ \hline
   沸点・融点&\multicolumn{1}{c}{低}&\multicolumn{2}{c}{\ce{<->}}&\multicolumn{1}{c|}{高}\\ \hline
   常温での状態&\hl{気体}&\hl{気体}&\hl{液体}&\hl{固体}\\ \hline
   色&\hl{淡黄}色&\hl{黄緑}色&\hl{赤褐}色&\hl{黒紫}色\\ \hline
   特徴&\hl{特異}臭&\hl{刺激}臭&揮発性&\hl{昇華}性\\ \hline
   \ce{H2}との反応&
   \begin{tabular}{c}
   \hl{冷暗所}でも\\爆発的に反応
   \end{tabular}&
   \begin{tabular}{c}
   \hl{常温}でも\hl{光}で\\爆発的に反応
   \end{tabular}&
   \begin{tabular}{c}
   \hl{加熱}して\\\hl{触媒}により反応
   \end{tabular}&
   \begin{tabular}{c}
   高温で平衡状態\\
   \hl{加熱}して\hl{触媒}により一部反応
   \end{tabular}\\ \hline
   水との反応&
   \begin{tabular}{c}
   水を酸化して酸素を発生\\
   \hl{激しく反応}
   \end{tabular}&
   \hl{一部とけて反応}&
   \hl{一部とけて反応}&
   \begin{tabular}{c}
   \hl{反応しない}\\\hl{\ce{KIaq}には可溶}
   \end{tabular}\\ \hline
  \end{tabular}
  }
  \end{center}
  
  \subsection{反応(単体)}
  \begin{itemize}
   \item フッ素と水素\\
   \shl{\ce{H2 + F2 ->T[常温で爆発的に反応] 2HF}}
   \item 塩素と水素\\
   \shl{\ce{H2 + Cl2 ->T[光を当てると爆発的に反応] 2HCl}}
   \item 臭素と水素\\
   \shl{\ce{H2 + Br2 ->T[高温で反応] 2HBr}}
   \item ヨウ素と水素\\
   \shl{\ce{H2 + I2 <=>T[高温で平衡] 2HI}}
   \item フッ素と水\\
   \shl{\ce{F2 + H2O ->T[常温で爆発的に反応] 2HF}}
   \item 塩素と水\\
   \shl{\ce{Cl2 + H2O ->T[光を当てると爆発的に反応] 2HCl}}
   \item 臭素と水\\
   \shl{\ce{Br2 + H2O ->T[高温で反応] 2HBr}}
   \item ヨウ素と水\\
   \shl{\ce{Br2 + H2O <=>T[高温で平衡] 2HI}}
   \item 臭化マグネシウムと塩素\\
   \ce{MgBr2 + 2HCl -> 2HBr + MgCl}
   \item ヨウ化カリウムと塩素\\
   \ce{2KI + Cl2 -> 2KCl + I2}
   \item 塩化カリウムと臭素\\
   \ce{2KCl + Br2 -> 2KBr + Cl2}
  \end{itemize}
  
  \subsection{フッ素 \ce{F}}
   \begin{itemize}
    \item 保存が困難
    \item \ce{Kr}や\ce{Xe}と反応
   \end{itemize}
   \subsubsection{製法}
   フッ化水素ナトリウム\ce{KHF2}のフッ化水素\ce{HF}溶液の電気分解 \K
  
  \subsection{塩素 \ce{Cl}}
   \hl{\ce{ClO-}}による\hl{殺菌・漂白}作用
   \subsubsection{製法}
    \begin{itemize}
     \item \hl{水酸化ナトリウム}の電気分解 \K\\
     \shl{\ce{2NaCl + 2H2O -> Cl2 + H2 + 2NaOH}}
     \item \hl{酸化マンガン(\ajRoman{3})}に\hl{濃硫酸}を加えて加熱\\
     \shl{\ce{NaCl + H2SO4 -> NaHSO4 + HCl ^}}
     \item \hl{高度さらし粉}と\hl{塩酸}\\
     \shl{\ce{CaCl(ClO)*H2O + 2HCl -> CaCl2 + Cl2 ^ + 2H2O}}
     \item \hl{さらし粉}と\hl{塩酸}\\
     \shl{\ce{CaCl(ClO)*H2O + 2HCl -> CaCl2 + Cl2 ^ + 2H2O}}
    \end{itemize}
   \subsubsection{塩素のオキソ酸}
    \begin{tabular}{r|lr}
     +\ajRoman{7}&\hl{\ce{HClO4}}&\hl{\quad 過塩素酸}\\
     +\ajRoman{5}&\hl{\ce{HClO3}}&\hl{\qquad 塩素酸}\\
     +\ajRoman{3}&\hl{\ce{HClO2}}&\hl{\quad 亜塩素酸}\\
     +\ajRoman{1}&\hl{\ce{HClO}}&\hl{次亜塩素酸}\\
    \end{tabular}
  \subsection{臭素 \ce{Br}}
  \ce{C=C}や\ce{C#C}の検出
  
  \subsection{ヨウ素 \ce{I}}
  \hl{ヨウ素デンプン}反応で\hl{青紫}色
 \newpage
 \twocolumn[
 \part{金属元素}
 ]
\end{document}