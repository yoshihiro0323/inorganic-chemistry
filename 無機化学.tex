\documentclass[dvipdfmx,a4paper]{jsarticle}
\usepackage[deluxe]{otf}
\usepackage[top=20truemm,bottom=20truemm,left=15truemm,right=15truemm]{geometry}
\usepackage{multirow}
\usepackage[version=3]{mhchem}
\usepackage{color}
\usepackage[dvipsnames]{xcolor}
\usepackage{enumerate}
\usepackage{tikz}
\usepackage{lscape}
\usepackage{amsmath}
\usepackage{chemfig}
\usepackage{fancyhdr}
\usepackage{lastpage}
\usepackage[dvipdfmx]{hyperref}
\usepackage{ascmac}
\usepackage{multicol}

\makeatletter
\def\section{\@startsection {section}{1}{\z@}{-2.5ex plus -1ex minus -.2ex}{2.5 ex plus .2ex}{\Large \bfseries \gtfamily}}
\def\subsection{\@startsection {subsection}{1}{\z@}{-1.5ex plus -1ex minus -.2ex}{2.3 ex plus .2ex}{\Large \gtfamily}}
\def\subsubsection{\@startsection {subsubsection}{1}{\z@}{-2.5ex plus -1ex minus -.2ex}{.3 ex plus .2ex}{\large \gtfamily}}
\makeatother

\newcommand{\hl}[1]{\underline{\textcolor{red}{\gtfamily #1}}}
\newcommand{\hce}[1]{\textcolor{blue}{\ce{#1}}}
\newcommand{\hlbox}[1]{\fbox{\textcolor{red}{\gtfamily #1}}}

\pagestyle{empty}

\newcommand{\stamp}[2]{%
  \begin{tikzpicture}[baseline=(A.base),font=\sffamily\small]
    \node[draw=#1, rectangle, rounded corners=2pt,
          text=#1,
          fill=#1!10!white,
          line width=.4pt, inner sep=1pt, outer sep=0pt
          ] (A) {#2};
  \end{tikzpicture}%
}
\newcommand{\K}{\stamp{blue}{工業的製法}}
\newcommand{\R}{\stamp{red}{例}}

\columnseprule=0.2mm

\pagestyle{fancy}
\lfoot{\hyperlink{top}{無機化学}}
\cfoot{\thepage/\pageref{LastPage}}

\hypersetup{
 bookmarks=false,
 colorlinks=true,
 linkcolor=black,
 citecolor=[rgb]{0,0.4,0.8},
 filecolor=black,
 urlcolor=[rgb]{0,0.4,0.8}
}

\begin{document}

%\setchemfig{debug=true}
%\setcharge{debug}

\thispagestyle{fancy}

\twocolumn
\twocolumn[
\begin{center}
\Huge{\gtfamily 無機化学}\\
\vspace{5mm}
\end{center}
]
\hypertarget{top}{\tableofcontents}
\twocolumn[
\newpage
 \part{非金属元素}
 ]
 \section{水素}
  \hl{無色無臭}の\hl{気体}\footnote{融点 14K 沸点 20K} \hl{最も軽く}、水に溶け\hl{にくい}
  \subsection{同位体}
  \ce{_{}^{1}H} 99\%以上 \ce{_{}^{2}H} (\hl{D})0.015\% \ce{_{}^{3}H} (\hl{T})微量
  \subsection{製法}
  \begin{itemize}
   \item ナフサの電気分解 \K
  \item 赤熱した\hl{コークス}に\hl{水蒸気}を吹き付ける \K\\
  \hce{C + H2O -> H2 + CO}
  \item \hl{水}(\hl{水酸化ナトリウム水溶液})の電気分解\\
  \hce{2H2O -> 2H2 + O2}
  \item \hl{イオン化傾向}が\hl{\ce{H2}より大きい}金属と希薄強酸\\
  \R \;\ce{Fe + 2HCl -> FeCl2 + H2 ^}\\
  \R \;\ce{Zn + 2HCl -> ZnCl2 + H2 ^}
 \end{itemize}
 \subsection{反応}
 \begin{itemize}
  \item 水素と酸素(爆鳴気の燃焼)\\
  \hce{2H2 + O2 -> H2O}
  \item 加熱した酸化銅(\ajRoman{2})と水素\\
  \hce{CuO + H2 -> Cu + H2O}
  \item 水酸化ナトリウムと水\\
  \hce{NaH + H2O -> NaOH + H2}
 \end{itemize}
 \newpage
 \section{貴ガス}
 \hl{\ce{He}}, \hl{\ce{Ne}}, \hl{\ce{Ar}}, \hl{\ce{Kr}}, \ce{Xe}, \ce{Rn}
  \subsection{性質}
  \begin{itemize}
   \item 無色・無臭
   \item 第18族元素であり、電子配置がオクテットを満たすため反応性が低い。
   \item イオン化エネルギーが極めて大きい。
   \item 電子親和力は\hl{極めて小さい}(\hl{ほぼ0})。
   \item 電気陰性度は\hl{定義されない}。
  \end{itemize}
  \subsection{生成}
  \ce{_{}^{40}K}の電子捕獲\\
  \quad\hce{_{}^{40}K + e- -> _{}^{40}Ar}
  \subsection{ヘリウム \ce{He}}
  浮揚ガス
  \subsection{ネオン \ce{Ne}}
  ネオンサイン
  \subsection{アルゴン \ce{Ar}}
  \ce{N2}, \ce{O2}に次いで3番目に空気中での存在量が多い(約1\%)。
  
 \newpage
 \onecolumn
 \section{ハロゲン}
  \subsection{単体}
  \subsubsection{性質}
  \begin{center}
  \begin{tabular}{|c||c|c|c|c|}\hline
   化学式&\ce{F2}&\ce{Cl2}&\ce{Br2}&\ce{I2}\\ \hline\hline
   分子量&\multicolumn{4}{|c|}{小 \quad\tikz \draw[line width=1pt,{Stealth}-{Stealth}] (0,0)--(10,0);\quad 大}\\ \hline
   分子間力&\multicolumn{4}{|c|}{弱 \quad\tikz \draw[line width=1pt,{Stealth}-{Stealth}] (0,0)--(10,0);\quad 強}\\ \hline
   反応性&\multicolumn{4}{|c|}{強 \quad\tikz \draw[line width=1pt,{Stealth}-{Stealth}] (0,0)--(10,0);\quad 弱}\\ \hline
   沸点・融点&\multicolumn{4}{|c|}{低 \quad\tikz \draw[line width=1pt,{Stealth}-{Stealth}] (0,0)--(10,0);\quad 高}\\ \hline
   常温での状態&\hl{気体}&\hl{気体}&\hl{液体}&\hl{固体}\\ \hline
   色&\hl{淡黄}色&\hl{黄緑}色&\hl{赤褐}色&\hl{黒紫}色\\ \hline
   特徴&\hl{特異}臭&\hl{刺激}臭&揮発性&\hl{昇華}性\\ \hline
   \ce{H2}との反応&
   \begin{tabular}{c}
   \hl{冷暗所}でも\\爆発的に反応
   \end{tabular}&
   \begin{tabular}{c}
   \hl{常温}でも\hl{光}で\\爆発的に反応
   \end{tabular}&
   \begin{tabular}{c}
   \hl{加熱}して\\\hl{触媒}により反応
   \end{tabular}&
   \begin{tabular}{c}
   高温で平衡状態\\
   \hl{加熱}して\hl{触媒}により一部反応
   \end{tabular}\\ \hline
   水との反応&
   \begin{tabular}{c}
   水を酸化して酸素を発生\\
   \hl{激しく反応}
   \end{tabular}&
   \hl{一部とけて反応}&
   \hl{一部とけて反応}&
   \begin{tabular}{c}
   \hl{反応しない}\\\hl{\ce{KIaq}には可溶}
   \end{tabular}\\ \hline
   用途&
   \begin{tabular}{c}
   保存が困難\\
   \ce{Kr}や\ce{Xe}と反応
   \end{tabular}&
   \begin{tabular}{c}
   \hl{\ce{ClO-}}による\\\hl{殺菌・漂白}作用
   \end{tabular}&
   \begin{tabular}{c}
   \ce{C=C}や\\
   \ce{C#C}の検出
   \end{tabular}&
   \begin{tabular}{c}
   \hl{ヨウ素デンプン}反応で\\
   \hl{青紫}色
   \end{tabular}\\ \hline
  \end{tabular}
  \end{center}
\begin{multicols}{2}
  \subsubsection{製法}
   \begin{itemize}
    \item フッ化水素ナトリウム\ce{KHF2}のフッ化水素\ce{HF}溶液の電気分解 \K\\
    \ce{KHF2 -> KF + HF}
    \item \hl{水酸化ナトリウム}の電気分解 \K\\
    \hce{2NaCl + 2H2O -> Cl2 + H2 + 2NaOH}
    \item \hl{酸化マンガン(\ajRoman{4})}に\hl{濃硫酸}を加えて加熱\\
    \hce{MnO2 + 4HCl ->[][\Delta] MnCl2 + Cl2 ^ + 2H2O}
    \item \hl{高度さらし粉}と\hl{塩酸}\\
    \hce{Ca(ClO)2*2H2O + 4HCl -> CaCl2 + 2Cl2 ^ + 4H2O}
    \item \hl{さらし粉}と\hl{塩酸}\\
    \hce{CaCl(ClO)*H2O + 2HCl -> CaCl2 + Cl2 ^ + 2H2O}
    \item 臭化マグネシウムと塩素\\
    \ce{MgBr2 + Cl2 -> MgCl2 + Br2}
    \item ヨウ化カリウムと塩素\\
    \ce{2KI + Cl2 -> 2KCl + I2}
   \end{itemize}
  \columnbreak
  \subsubsection{反応}
  \begin{itemize}
   \item フッ素と水素\\
   \hce{H2 + F2 ->T[常温で爆発的に反応] 2HF}
   \item 塩素と水素\\
   \hce{H2 + Cl2 ->T[光を当てると爆発的に反応] 2HCl}
   \item 臭素と水素\\
   \hce{H2 + Br2 ->T[高温で反応] 2HBr}
   \item ヨウ素と水素\\
   \hce{H2 + I2 <=>T[高温で平衡] 2HI}
   \item フッ素と水\\
   \hce{2F2 + 2H2O -> 4HF + O2}
   \item 塩素と水\\
   \hce{Cl2 + H2O <=> HCl + HClO}
   \item 臭素と水\\
   \hce{Br2 + H2O <=> HBr + HBrO}
   \item ヨウ素の固体がヨウ化物イオン存在下で三ヨウ化物イオンを形成して溶解する反応\\
   \hce{I2 + I- -> I3-}
  \end{itemize}
\end{multicols}
\newpage
  \subsubsection{塩素発生実験の装置}
  \ce{MnO2 + 4HCl ->[][\Delta] MnCl2 + Cl2 ^ + 2H2O}
  \ce{Cl2},\ce{HCl},\ce{H2O}\\
  ↓\hl{\quad 水\quad}に通す (\ce{HCl}の除去)\\
  \ce{Cl2},\ce{H2O}\\
  ↓\hl{濃硫酸}に通す (\ce{H2O}の除去)\\
  \ce{Cl2}
  \subsubsection{塩素のオキソ酸}
  オキソ酸$\cdots$\hl{酸素を含む酸性物質}\\\\
   \begin{tabular}{r|lrl}
    +\ajRoman{7}&\hl{\ce{HClO4}}&\hl{\quad 過塩素酸}&\hlbox{\chemfig{H-[,0.6]O-[,0.6]Cl(-[:90,0.6]O)(-[:-90,0.6]O)-[,0.6]O}}\\\hline
    +\ajRoman{5}&\hl{\ce{HClO3}}&\hl{\qquad 塩素酸}&\hlbox{\chemfig{H-[,0.6]O-[,0.6]Cl(-[:90,0.6]O)-[,0.6]O}}\\\hline
    +\ajRoman{3}&\hl{\ce{HClO2}}&\hl{\quad 亜塩素酸}&\hlbox{\chemfig{H-[,0.6]O-[,0.6]Cl-[,0.6]O}}\\\hline
    +\ajRoman{1}&\hl{\ce{HClO}}&\hl{次亜塩素酸}&\hlbox{\chemfig{H-[,0.6]O-[,0.6]Cl}}\\
   \end{tabular}
  \subsection{ハロゲン化水素}
   \subsubsection{性質}
   \begin{table}[hbtp]
   \begin{tabular}{|c||c|c|c|c|}\hline
   化学式&\ce{HF}&\ce{HCl}&\ce{HBr}&\ce{HI}\\ \hline
   色・臭い&\multicolumn{4}{|c|}{\hl{無}色\hl{刺激}臭}\\ \hline
   沸点&$20^\circ$C&$-85^\circ$C&$-67^\circ$C&$-35^\circ$C\\ \hline
   水との反応&\multicolumn{4}{|c|}{\hl{よく溶ける}}\\ \hline
   水溶液&\multicolumn{1}{c}{\hl{フッ化水素酸}}&\multicolumn{1}{c}{\hl{塩酸}}&\multicolumn{1}{c}{\hl{臭化水素酸}}&\hl{ヨウ化水素酸}\\
   (強弱)&
   \multicolumn{4}{|c|}{
   \begin{tabular}{ccccccc}
   \hl{弱酸}&$\ll$&\hl{強酸}&$<$&\hl{強酸}&$<$&\hl{強酸}
   \end{tabular}
   }\\ \hline
   用途&
   \begin{tabular}{c}
   \hl{ガラス}と反応\\
   $\Rightarrow$ ポリエチレン瓶
   \end{tabular}&
   \begin{tabular}{c}
   \hl{アンモニア}の検出\\
   各種工業
   \end{tabular}&
   半導体加工&
   \begin{tabular}{c}
   インジウムスズ\\
   酸化物の加工
   \end{tabular}
   \\ \hline
   \end{tabular}
   \end{table}
   \subsubsection{製法}
    \begin{itemize}
     \item \hl{ホタル石}に\hl{濃硫酸}を加えて加熱(\hl{弱酸遊離})\\
     \hce{CaF2 + H2SO4 ->[][\Delta] CaSO4 + 2HF ^}
     \item \hl{水素}と\hl{塩素} \K\\
     \hce{H2 + Cl2 -> 2HCl ^}
     \item \hl{塩化ナトリウム}に\hl{濃硫酸}に加えて加熱(\hl{弱酸}酸・\hl{揮発性}酸の追い出し)\\
     \hce{NaCl + H2SO4 ->[][\Delta] NaHSO4 + HCl ^}
    \end{itemize}
   \subsubsection{反応}
   \begin{itemize}
    \item 気体のフッ化水素がガラスを侵食する反応\\
    \hce{SiO2 + 4HF(g) -> SiF4 ^ + 2H2O}
    \item フッ化水素酸(水溶液)がガラスを侵食する反応\\
    \hce{SiO2 + 6HF(aq) -> H2SiF6 ^ + 2H2O}
    \item \hl{塩化水素}による\hl{アンモニア}の検出\\
    \hce{AgO2 + 2HF -> 2AgF + H2O}
   \end{itemize}
  \subsection{ハロゲン化銀}
  \subsubsection{性質}
  \begin{tabular}{|c||c|c|c|c|}\hline
  化学式&\ce{AgF}&\ce{AgCl}&\ce{AgBr}&\ce{AgI}\\ \hline
  固体の色&\hl{黄褐}色&\hl{白}色&\hl{淡黄}色&\hl{黄}色\\ \hline
  水との反応&\hl{よく溶ける}&\multicolumn{3}{|c|}{\hl{ほとんど溶けない}}\\ \hline
  光との反応&\hl{感光}&\multicolumn{3}{|c|}{感光性(→\hl{\ce{Ag}})}\\ \hline
  \end{tabular}\\
  \subsubsection{製法}
  \begin{itemize}
   \item 酸化銀(\ajRoman{1})にフッ化水素酸を加えて蒸発圧縮\\
   \hce{Ag2O + 2HF -> 2AgF + H2O}
   \item ハロゲン化水素イオンを含む水溶液と\hl{硝酸銀水溶液}\\
   \hce{Ag+ + X- -> AgX v}
  \end{itemize}
  \subsection{次亜塩素酸塩}
  \subsubsection{性質}
  \hl{酸化}剤として反応(\hl{殺菌}・\hl{漂白}作用\\
  \quad \hce{ClO- + 2H+ + 2e- -> H2O + Cl-}
  \subsubsection{製法}
  \begin{itemize}
   \item 水酸化ナトリウム水溶液と塩素\\
   \hce{2NaOH + Cl2 -> NaCl + NaClO + H2O}
   \item 水酸化カルシウムと塩素\\
   \hce{Ca(OH)2 + Cl2 -> CaCl(ClO)*H2O}
  \end{itemize}
  \subsection{水素酸カリウム}
  化学式:\hl{\ce{KClO3}}
  \subsubsection{性質}
  \hl{酸素}の生成(\hl{二酸化マンガン}を触媒に加熱)\\
  \quad \hce{2KClO3 ->C[MnO2][\Delta] 2KClO + 3O2 ^}
  
 \newpage
 \section{酸素}
  \subsection{酸素原子}
  同\hl{位}体:酸素(\ce{O2}),\hl{オゾン}(\ce{O3})\\
  \quad 地球の地殻に\hl{最も多く}存在
   \begin{itembox}[l]{地球の地殻における元素の存在率}
   \begin{tabular}{ccccccccccc}
   \hl{\ce{O}} & \multirow{2}{*}{$>$} & \hl{\ce{Si}} & \multirow{2}{*}{$>$} & \hl{\ce{Al}} & \multirow{2}{*}{$>$} & \hl{\ce{Fe}} & \multirow{2}{*}{$>$} & \hl{\ce{Ca}} & \multirow{2}{*}{$>$} & \hl{\ce{Na}}\\
   \hl{酸素}&&\hl{ケイ素}&&\hl{アルミニウム}&&\hl{鉄}&&\hl{カルシウム}&&\hl{ナトリウム}\\
   46.6\%&&27.7\%&&8.13\%&&5.00\%&&3.63\%&&2.83\% \\
   おっ&&し&&ゃる&&て&&か&&な
   \end{tabular}
   \end{itembox}
   
 \begin{multicols}{2}
  \subsection{酸素}
   化学式:\ce{O2}
   \subsubsection{性質}
   \begin{itemize}
   \item \hl{無}色\hl{無}臭の\hl{気体}
   \item 沸点 $-183^\circ$C
   \end{itemize}
   \subsubsection{製法}
   \begin{itemize}
    \item \hl{液体空気}の\hl{分留} \K
    \item \hl{水}(\hl{水酸化ナトリウム水溶液})の\hl{電気分解}\\
    \hce{2H2O -> 2H2 ^ + O2 ^}
    \item \hl{過酸化水素水}(\hl{オキシドール})の分解\\
    \hce{2H2O2 ->C[MnO2] O2 ^ + 2H2O}
    \item \hl{塩素酸カリウム}の熱分解\\
    \hce{2KClO3 ->C[MnO2][\Delta] 2KClO + 3O2 ^}
   \end{itemize}
   \subsubsection{反応}
   \hl{酸化}剤としての反応\\
   \quad \hce{O2 + 4H+ + 4e- -> 2H2O}
  \subsection{オゾン}
   化学式:\hl{\ce{O3}}
   \subsubsection{性質}
   \begin{itemize}
    \item \hl{ニンニク}臭(\hl{特異}臭)を持つ\hl{淡青}色の\hl{気体}(常温)
    \item 水に\hl{少し溶ける}
    \item \hl{殺菌}・\hl{脱臭}作用
   \end{itemize}
   \columnbreak
   \begin{itembox}[l]{オゾンにおける酸素原子の運動}
   \begin{center}
   \chemfig{\charge{[circle]30=\.,150=\:,270=\:}{O}-[:30,0.6,,,opacity = 0]\charge{[circle]90=\:,210=\.,330=\:}{O}=[:-30,0.6,,,opacity = 0]\charge{[circle]30=\:,150=\:,270=\:}{O}}
   \ce{<=>T[\hl{振動}]} 
   \chemfig{\charge{[circle]30=\:,150=\:,270=\:}{O}-[:30,0.6,,,opacity = 0]\charge{[circle]90=\:,210=\:,330=\.}{O}=[:-30,0.6,,,opacity = 0]\charge{[circle]30=\:,150=\.,270=\:}{O}}
   \end{center}
   \hfill $\Rightarrow$\hl{極性}
   \end{itembox}
   \subsubsection{製法}
   酸素中で\hl{無声放電}/強い\hl{紫外線}を当てる\\
   \quad \hce{3O2 -> 2O3}
   \subsubsection{反応}
   \begin{itemize}
   \item \hl{酸化}剤としての反応\\
   \hce{O3 + 2H+ + 2e- -> O2 + H2O}
   \item 湿らせた\hl{ヨウ化カリウムでんぷん紙}を\hl{青}色に変色\\
   \hce{O3 + 2KI + H2O -> I2 + O2 + 2KOH}
   \item 酸化カルシウムと水\\
   \hce{CaO + H2O -> Ca(OH)2}
   \item 二酸化窒素と水\\
   \hce{3NO2 + H2O -> 2HNO3 + NO}
   \item 酸化銅(\ajRoman{2})と塩化水素\\
   \hce{CuO + 2HCl -> CuCl2 + H2O}
   \item 酸化アルミニウムと硫酸\\
   \hce{Al2O3 + 3H2SO4 -> Al2(SO4)3 + 3H2O}
   \item 酸化アルミニウムと水酸化ナトリウム水溶液\\
   \hce{Al2O3 + 2NaOH -> 3H2O ->2Na[Al(OH)+]}
   \end{itemize}
 \end{multicols}
  \subsection{酸化物の分類}
  \begin{tabular}{|c|c|c|c|}\hline
  &塩基性酸化物&両性酸化物&酸性酸化物\\ \hline
  元素&\hl{陽性}の\hl{大き}い\hl{金属}元素&\hl{陽性}の\hl{小さ}い\hl{金属}元素&\hl{非金属元素}\\ \hline
  水との反応&\hl{塩基性}&\hl{ほとんど溶けない}&\hl{酸性}($\Rightarrow$\hl{オキソ酸})\\ \hline
  中和&\hl{酸}と反応&\hl{酸・塩基}と反応&\hl{塩基}と反応\\ \hline
  \end{tabular}\\
  両性酸化物$\cdots$\hl{アルミニウム}(\hl{Al}),\hl{亜鉛}(\hl{Zn}),\hl{スズ}(\hl{Sn}),\hl{鉛}(\hl{Pb})\footnote{覚え方:ああすんなり}
  \begin{enumerate}[\R]
   \item \ce{CO2 + H2O -> H2CO3}
   \item \ce{SO2 + H2O -> H2SO3}
   \item \ce{3NO2 + H2O -> 2HNO3 + NO}
  \end{enumerate}
  \subsection{水の特異性}
  \begin{multicols}{2}
  \begin{itemize}
   \item \hl{極性}分子
   \item 周りの4つの分子と\hl{水素}結合
   \item 異常に\hl{高い}沸点
   \item \hl{隙間の多い}結晶構造(密度:固体\hl{$<$}液体)
   \item 特異な\hl{融解曲線}
  \end{itemize}
  \columnbreak
  \chemfig{H(-[:190,0.7,,,black,line width=2pt,dash pattern=on 1pt off 2pt]O(-[::-40,0.7]H)-[::40,0.7]H)-[:10]O(-[:90,0.9,,,black,line width=2pt,dash pattern=on 1pt off 2pt]H-[:120,0.7]O-[:200,0.7]H)(-[:-10,,,,black,line width=2pt,dash pattern=on 1pt off 2pt]H-[:40,0.7]O-[:-40,0.7]H)-[:-60,0.9]H-[:-60,0.7,,,black,line width=2pt,dash pattern=on 1pt off 2pt]O(-[,0.7]H)-[:-80,0.7]H}
  \end{multicols}
\twocolumn
 \section{硫黄}
  \subsection{硫黄}
   \subsubsection{性質}
   \begin{tabular}{|c|c|c|c|}\hline
   &\hl{斜方硫黄}&\hl{単斜硫黄}&\hl{ゴム状硫黄}\\ \hline
   化学式&\hl{\ce{S8}}&\hl{\ce{S8}}&\hl{\ce{S_{$x$}}}\\ \hline
   色&\hl{黄}色&\hl{黄}色&\hl{黄}色\\ \hline
   構造&\hl{塊状}結晶&\hl{針状}結晶&\hl{不定形}固体\\ \hline
   融点&$113^\circ$C&$119^\circ$C&不定\\ \hline
   構造&\multicolumn{2}{|c|}{\chemfig{S?-[:-30,0.57735]S-[:30,0.57735]S-[:-30,0.57735]S-[:30,0.57735]S-[:-120]S-[:135,0.7071]S-[:-135,0.7071]S?}}&\chemfig{-[:30,0.5,,,black,line width=2pt,dash pattern=on 1pt off 2pt]S-[:30,0.7]S-[::-90,0.7]S-[::100,0.7]S-[::-100,0.7]S-[::-60,0.7]S-[::-60,0.7]S-[::0,0.5,,,black,line width=2pt,dash pattern=on 1pt off 2pt]}\\ \hline
   \ce{CS2}との反応&\hl{溶ける}&\hl{溶ける}&\hl{溶けない}\\ \hline
   \end{tabular}\\
   \ce{CS2}$\cdots$無色・芳香性・揮発性$\Rightarrow$\hl{無極性}触媒
   \subsubsection{反応}
   \begin{itemize}
    \item 高温で多くの金属(\ce{Au}、\ce{Pt}を除く)との反応\\
    \hce{Fe + S -> FeS}
    \item 空気中で\hl{青}色の炎を上げて燃焼\\
    \hce{S + O2 -> SO2}
   \end{itemize}
  \subsection{硫化水素}
   化学式:\hl{\ce{H2S}}
   \subsubsection{性質}
   \begin{itemize}
    \item \hl{無}色・\hl{腐卵}臭
    \item \hl{弱酸}性\\$\left\{
    \begin{tabular}[h]{ll}
    \hl{\ce{H2S <=> H+ + HS-}}& $K_{1}=9.5\times10^{-8}$ mol/L\\
    \hl{\ce{HS- <=> H+ + S-}}& $K_{2}=1.3\times10^{-14}$ mol/L
    \end{tabular}
    \right.$
   \end{itemize}
   \subsubsection{製法}
   \begin{itemize}
    \item 酸化鉄(\ajRoman{2})と希塩酸\\
    \hce{FeS + 2HCl -> FeCl2 + H2S ^}
    \item 酸化鉄(\ajRoman{2})と希硫酸\\
    \hce{FeS + H2SO4 -> FeSO4 + H2S ^}
   \end{itemize}
   \subsubsection{反応}
  \subsection{二酸化硫黄(亜硫酸ガス)}
  化学式:\hl{\ce{SO2}}
  電子式:\quad\hlbox{\chemfig{\charge{[circle]0=\:,90=\:,180=\:,270=\:}{O}-[-2,0.5,,,opacity=0]\charge{[circle]0=\:,-90=\:}{S}-[,0.6,,,opacity=0]\charge{[circle]90=\:,180=\:,270=\:}{O}}}
  \subsubsection{性質}
  \begin{itemize}
   \item \hl{無}色、\hl{刺激}臭の\hl{気体}
   \item 水に\hl{溶けやすい}
   \item \hl{弱酸}性\\
    \begin{tabular}{ll}
    \hl{\ce{H2O + SO2 <=> H+ + HSO3-}}& $K_{1}=1.4\times10^{-2}$ mol/L\\
    \end{tabular}
   \item \hl{還元}剤(\hl{漂白}作用)\\
   \hce{SO2 + 2H2O -> SO4^2- + 4H+ + 2e-}
   \item \hl{酸化}剤(\hl{\ce{H2S}}などの強い還元剤に対して)\\
   \hce{SO2 + 4H+ + 4e- -> S + 2H2O}
  \end{itemize}
  \subsubsection{製法}
  \begin{itemize}
   \item 硫黄や硫化物の\hl{燃焼} \K\\
   \hce{2H2S + 3O2 -> 2SO2 + 2H2O}
   \item \hl{亜硫酸ナトリウム}と{希硫酸}\\
   \hce{Na2SO3 + H2SO4 ->[][\Delta] NaHSO4 + SO2 ^ + H2O}
   \item \hl{銅}と\hl{熱濃硫酸}\\
   \hce{Cu + 2H2SO4 -> CuSO4 + SO2 ^ + 2H2O}
  \end{itemize}
  \subsubsection{反応}
  \begin{itemize}
   \item 二酸化硫黄の水への溶解\\
   \hce{SO2 + H2O -> H2SO3}
   \item 二酸化硫黄と硫化水素\\
   \hce{SO2 + 2H2S -> 3S + 3H2O}
   \item 硫酸酸性で過マンガン酸カリウムと二酸化硫黄\\
   \hce{2KMnO4 + 5SO2 + 2H2O -> 2MnSO4 + 2H2SO4 + K2SO4}
  \end{itemize}
  \subsection{硫酸}
  \subsubsection{性質}
  \begin{itemize}
   \item \hl{無}色\hl{無}臭の\hl{液体}
   \item 水に\hl{非常によく溶ける}
   \item 溶解熱が\hl{非常に大きい}
   \item \hl{水に濃硫酸}を加えて希釈
   \item \hl{不揮発}性で密度が\hl{大き}く、\hl{粘度}が大きい \stamp{purple}{濃硫酸}
   \item \hl{吸湿性}・\hl{脱水作用} \stamp{purple}{濃硫酸}
   \item \hl{強酸性} \stamp{magenta}{希硫酸}\\
   $\left(
   \begin{tabular}{cc}
   \hl{\ce{H2SO4 <=> H+ + HSO4-}}& $K_{1}>10^8$mol/L
   \end{tabular}
   \right)$
   \item \hl{弱酸性} \stamp{purple}{濃硫酸} (\hl{水が少なく}、\hl{\ce{H3O+}}の濃度が小さい)
   \item \hl{酸化}剤として働く \stamp{violet}{熱濃硫酸}\\
   \hl{\ce{H2SO4 + 2H+ + 2e- -> SO4 + 2H2O}}
   \item \hl{アルカリ性土類金属}(\hl{\ce{Ca}},\hl{\ce{Be}})、\hl{\ce{Pb}}と難容性の塩を生成 \stamp{magenta}{希硫酸}
  \end{itemize}
  \subsubsection{製法}
  \begin{itembox}[l]{\hl{接触}法 \K}
  \begin{enumerate}
   \item 黄鉄鉱\ce{FeS2}の燃焼\\
   \hce{4FeS2 + 11O2 -> 2Fe2O3 + 8SO2}\\
   $\left(\hce{S + O2 -> SO2}\right)$
   \item \hl{酸化バナジウム}触媒で酸化\\
   \hce{2SO2 + O2 ->C[V2O5] 2SO3}
   \item \hl{濃硫酸}に吸収させて\hl{発煙硫酸}とした後、希硫酸を加えて希釈\\
   \hce{SO3 + H2O -> H2SO4}
  \end{enumerate}
  \end{itembox}
  \subsubsection{反応}
  \begin{itemize}
   \item 硝酸カリウムに濃硫酸を加えて加熱\\
   \hce{KNO3 + H2SO4 -> HNO3 + KHSO4}
   \item スクロースと濃硫酸\\
   \hce{C12H22O11 ->C[H2SO4] 12C + 11H2O}
   \item 希硫酸と水酸化ナトリウム\\
   \hce{H2SO4 + 2NaOH -> Na2SO4 + 2H2O}
   \item 銀と熱濃硫酸\\
   \hce{2Ag + 2H2SO4 -> Ag2SO4 + SO2 + 2H2O}
   \item 塩化バリウム水溶液と希硫酸\\
   \hce{BaCl2 + H2SO4 -> BaSO4 v + 2HCl}
  \end{itemize}
  \subsection{チオ硫酸ナトリウム(ハイポ)}
  化学式:\hl{\ce{Na2S2O3}}\\\\
  電子式:\chemfig{H-[,0.5,,,opacity=0]\charge{[circle]0=\:,90=\:,180=\:,270=\:}{O}-[,0.5,,,opacity=0]S(-[2,0.5,,,opacity=0]\charge{[circle]0=\:,90=\:,180=\:,270=\:}{O})(-[-2,0.5,,,opacity=0]\charge{[circle]0=\:,90=\:,180=\:,270=\:}{O})-[,0.5,,,opacity=0]\charge{[circle]0=\:,90=\:,180=\:,270=\:}{O}-[,0.5,,,opacity=0]H}\\
  \begin{tabular}{cc}
  \chemfig{O^{-}-[,0.8]S(-[2,0.7,,,-latex]O)(-[-2,0.7,,,-latex]O)-[,0.8]O^{-}}&\chemfig{S^{-}-[,0.8]S(-[2,0.7,,,-latex]O)(-[-2,0.7,,,-latex]O)-[,0.8]O^{-}}\\
  \hl{硫酸}イオン&\hl{チオ硫酸}イオン
  \end{tabular}
  \subsubsection{性質}
   \begin{itemize}
    \item 無色透明の結晶(5水和物)で、水に溶けやすい。
    \item \hl{還元}剤として反応\\
    \R 水道水の脱塩素剤(カルキ抜き)\\
    \hl{\ce{2S2O3^2- -> S4O6 + 2e-}}\\\\
\chemfig{\charge{[circle]0=\:,90=\:,180=\:,270=\:,135[circle,anchor=180+\chargeangle]={\tiny $\ominus$}}{O}-[,0.5,,,opacity=0]S(-[2,0.5,,,opacity=0]\charge{[circle]0=\:,90=\:,180=\:,270=\:}{O})(-[-2,0.5,,,opacity=0]\charge{[circle]0=\:,90=\:,180=\:,270=\:}{O})-[,0.5,,,opacity=0]\charge{[circle]0=\:,90=\:,180=\:,270=\:,45[circle,anchor=180+\chargeangle]={\tiny $\ominus$}}{S}}
  \quad
\chemfig{\charge{[circle]0=\:,90=\:,180=\:,270=\:,135[circle,anchor=180+\chargeangle]={\tiny $\ominus$}}{S}-[,0.5,,,opacity=0]S(-[2,0.5,,,opacity=0]\charge{[circle]0=\:,90=\:,180=\:,270=\:}{O})(-[-2,0.5,,,opacity=0]\charge{[circle]0=\:,90=\:,180=\:,270=\:}{O})-[,0.5,,,opacity=0]\charge{[circle]0=\:,90=\:,180=\:,270=\:,45[circle,anchor=180+\chargeangle]={\tiny $\ominus$}}{O}}
 \quad \\ \hfill \tikz \draw[->] (0,0)--(0.7,0); \quad
\chemfig{\charge{[circle]0=\:,90=\:,180=\:,270=\:,135[circle,anchor=180+\chargeangle]={\tiny $\ominus$}}{O}-[,0.5,,,opacity=0]S(-[2,0.5,,,opacity=0]\charge{[circle]0=\:,90=\:,180=\:,270=\:}{O})(-[-2,0.5,,,opacity=0]\charge{[circle]0=\:,90=\:,180=\:,270=\:}{O})-[,0.5,,,opacity=0]\charge{[circle]0=\:,90=\:,180=\:,270=\:}{S}-[,0.5,,,opacity=0]\charge{[circle]0=\:,90=\:,270=\:}{S}-[,0.5,,,opacity=0]S(-[2,0.5,,,opacity=0]\charge{[circle]0=\:,90=\:,180=\:,270=\:}{O})(-[-2,0.5,,,opacity=0]\charge{[circle]0=\:,90=\:,180=\:,270=\:}{O})-[,0.5,,,opacity=0]\charge{[circle]0=\:,90=\:,180=\:,270=\:,45[circle,anchor=180+\chargeangle]={\tiny $\ominus$}}{O}} \quad+\ce{2e-}
  \end{itemize}
  \subsubsection{製法}
  亜硫酸ナトリウム水溶液に硫黄を加えて加熱\\
  \hce{Na2SO4 + S_{$n$} -> Na2S2O3}
  \subsubsection{反応}
  ヨウ素とチオ硫酸ナトリウム\\
  \hce{I2 + 2Na2S2O3 -> 2NaI + Na2S4O6}
  \twocolumn[
  \subsection{重金属の硫化物}
  \begin{tabular}{|cccccc|cccc|} \hline
  \multicolumn{6}{|c|}{酸性でも沈澱(全液性で沈澱)}&\multicolumn{4}{|c|}{中性・塩基性で沈澱(酸性では溶解)}\\ \hline
  \ce{Ag2S}&\ce{HgS}&\ce{CuS}&\ce{PbS}&\ce{SnS}&\ce{CdS}&\ce{NiS}&\ce{FeS}&\ce{ZnS}&\ce{MnS}\\
  \hl{黒}色&\hl{黒}色&\hl{黒}色&\hl{黒}色&\hl{褐}色&\hl{黒}色&\hl{黒}色&\hl{黒}色&\hl{白}色&\hl{淡赤}色\\ \hline
  \multicolumn{10}{c}{
  \begin{tabular}{ccc}
  \hl{低}&イオン化傾向&\hl{高}\\
  \hl{極小}&塩の溶解度積($K_{sp}$)&\hl{小}
  \end{tabular}
  }
  \end{tabular}
  ]
 \section{窒素}
 \subsection{窒素}
 化学式:\ce{N2}
  \subsubsection{性質}
   \begin{itemize}
    \item \hl{無色無臭}の\hl{気体}
    \item 空気の$78\%$を占める
    \item 水に溶け\hl{にくい}(\hl{無極性}分子)
    \item 常温で\hl{不活性}(食品などの\hl{酸化防止})
    \item 高エネルギー状態(\hl{高温}・\hl{放電})では反応
   \end{itemize}
  \subsubsection{製法}
   \begin{itemize}
    \item \hl{液体窒素の分留} \K
    \item \hl{亜硝酸アンモニウム}の\hl{熱分解}\\
    \ce{NH4NO2 -> N2 + 2H2O}
   \end{itemize}
  \subsubsection{反応}
   \begin{itemize}
    \item 窒素と酸素\\
    \hce{N2 + 2O2 -> 2NO2}
    $\left\{
    \begin{tabular}{r}
    \hce{N2 + O2 -> 2NO}\phantom{${}_{2}$}\\
    \hce{2NO + O2 -> 2NO2}
    \end{tabular}
    \right.$
    \item 窒素とマグネシウム\\
    \hce{3Mg + N2 -> Mg3N2}
   \end{itemize}
 \subsection{アンモニア}
 化学式:\hl{\ce{NH3}}
  \subsubsection{性質}
  \begin{itemize}
   \item \hl{無}色\hl{刺激}臭の\hl{気体}
   \item \hl{水素}結合
   \item 水に\hl{非常によく溶ける}(\hl{上方置換})
   \item \hl{塩基}性\\
   $\left(
   \begin{tabular}{l}
    \hl{\ce{NH3 + H2O <=> NH4+ + OH-}}\\
    $K_{1}=1.7\times10^{-5}$ mol/L\\
   \end{tabular}
   \right)$
   \item \hl{塩素}の検出
   \item 高温・高圧で二酸化炭素と反応して、\hl{尿素}を生成
  \end{itemize}
  \subsubsection{製法}
  \begin{itemize}
   \item \hl{ハーバーボッシュ法} \K\\
   \hl{低}温\hl{高}圧で、\hl{四酸化三鉄}(\hl{\ce{Fe3O4}})触媒\\
   \hce{N2 + 3H2 <=> 2NH3}
   \item \hl{塩化アンモニウム}と\hl{水酸化カルシウム}を混ぜて加熱\\
   \hce{2NH4Cl + Ca(OH)2 -> 2NH3 ^ + CaCl2 +2H2O}
  \end{itemize}
  \subsubsection{反応}
  \begin{itemize}
   \item 硫酸とアンモニア\\
   \hce{2NH3 + H2SO4 -> (NH4)2SO4}
   \item 塩素の検出\\
   \hce{NH3 + HCl -> NH4Cl v}
   \item アンモニアと二酸化炭素\\
   \hce{2NH3 + CO2 -> (NH2)2CO + H2O}
  \end{itemize}
  \subsection{一酸化二窒素(笑気ガス)}
  化学式:\hl{\ce{N2O}}
  \subsection{性質}
  \begin{itemize}
   \item 無色、少し甘味のある気体
   \item 水に少し溶ける
   \item 常温では反応性が低い
   \item \hl{麻酔}効果
  \end{itemize}
  \subsection{製法}
  \hl{硝酸アンモニウム}の熱分解\\
  \hce{NH4NO2 ->[][\Delta] N2O + 2H2O}
  \subsection{一酸化窒素}
  化学式:\hl{\ce{NO}}
  \subsubsection{性質}
  \begin{itemize}
   \item \hl{無}色\hl{無}臭の\hl{気体}
   \item 中性で水に溶けにくい
   \item 空気中では\hl{酸素}とすぐに反応
   \item 血管拡張作用・神経伝達物質
  \end{itemize}
  \subsubsection{製法}
  \hl{銅}と\hl{希硝酸}\\
  \hce{3Cu + 8HNO3 -> 3Cu(NO3)2 + 2NO + 4H2O}
  \subsubsection{反応}
  酸素と反応\\
  \hce{2NO + O2 -> 2NO2}
  \subsection{二酸化窒素}
  化学式:\hl{\ce{NO2}}
  \subsubsection{性質}
  \begin{itemize}
   \item \hl{赤褐}色\hl{刺激}臭の\hl{気体}
   \item 水と反応して\hl{強酸}性(\hl{酸性雨}の原因)
   \item 常温では\hl{四酸化二窒素}(\hl{無}色)と\hl{平衡状態}\\
   \hce{2NO2 <=> N2O4}
   \item $140^{\circ}$C以上で熱分解\\
   \hce{2NO2 -> 2NO + O2}
  \end{itemize}
  \subsubsection{製法}
  \hl{銅}と\hl{濃硝酸}\\
  \hce{Cu + 4HNO3 -> Cu(NO3)2 + 2NO2 + 2H2O}
  \subsection{硝酸}
  化学式:\hl{\ce{HNO3}}
  \subsubsection{性質}
  \begin{itemize}
   \item \hl{}色\hl{}臭で\hl{}の\hl{}
   \item 水に\hl{}
  \end{itemize}
  \subsubsection{製法}
  \begin{itemize}
   \item \hl{オストワルト法}\\
   \hce{NH3 + 2O2 -> HNO3 + H2O}
   \begin{enumerate}
    \item \hl{白金}触媒で\hl{アンモニア}を\hl{酸化}\\
    \hce{4NH3 + 5O2 -> 4NO + 6H2O}
    \item \hl{空気酸化}\\
    \hce{2NO + O2 -> 2NO2}
    \item \hl{水}と反応\\
    \hce{3NO2 + H2O -> 2HNO3 + NO}
   \end{enumerate}
   \item \hl{硝酸塩}に\hl{濃硫酸}を加えて加熱\\
   
  \end{itemize}
  \subsubsection{反応}
  
 \twocolumn
 \twocolumn[
 \part{金属元素}]
 \onecolumn
 \part{APPENDIX}
  \section{気体の乾燥剤}
  固体の乾燥剤は\hl{U字管}につめて、液体の乾燥剤は\hl{洗気瓶}に入れて使用。\\
  \begin{tabular}{|c|cc|c|c|}\hline
  性質&乾燥剤&化学式&対象&対象外(不適)\\ \hline
  \multirow{2}{*}{酸性}&\hl{十酸化四リン}&\hl{\ce{P4O10}}&\multirow{2}{*}{酸性・中性}&塩基性の気体(\hl{\ce{NH3}})\\ \cline{2-3} \cline{5-5}
  &\hl{濃硫酸}&\hl{\ce{H2SO4}}&&+\hl{\ce{H2S}}(\hl{還元剤})\\ \hline
  \multirow{2}{*}{中性}&\hl{塩化カルシウム}&\hl{\ce{CaCl2}}&\multirow{2}{*}{ほとんど全て}&\hl{\ce{NH3}}\\ \cline{2-3} \cline{5-5}
  &\hl{シリカゲル}&\hl{\ce{SiO2*$n$H2O}}&&特になし\\ \hline
  \multirow{2}{*}{塩基性}&\hl{酸化カルシウム}&\hl{\ce{CaO}}&\multirow{2}{*}{中性・塩基性}&酸性の気体\\ \cline{2-3}
  &\hl{ソーダ石灰}&\hl{\ce{CaO}と\ce{NaOH}}&&\hl{\ce{Cl2}},\hl{\ce{HCl}},\hl{\ce{H2S}},\hl{\ce{SO2}},\hl{\ce{CO2}},\hl{\ce{NO2}}\\ \hline
  \end{tabular}
\end{document}