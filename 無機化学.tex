\documentclass[dvipdfmx,a4paper]{jsarticle}
\usepackage[top=20truemm,bottom=20truemm,left=20truemm,right=20truemm]{geometry}
\usepackage{multirow}
\usepackage[version=3]{mhchem}
\usepackage{color}
\usepackage{enumerate}
\usepackage{tikz}
\usepackage{lscape}
\usepackage{amsmath}
\usepackage{otf}
\usepackage{chemfig}
\usepackage{fancyhdr}
\usepackage{lastpage}
\usepackage[dvipdfmx]{hyperref}
\usepackage{ascmac}

\newcommand{\hl}[1]{\underline{\textcolor{red}{\gtfamily #1}}}
\newcommand{\hce}[1]{{\textcolor{blue}{\ce{#1}}}}
\newcommand{\hlbox}[1]{\fbox{\textcolor{red}{\gtfamily #1}}}

\pagestyle{empty}

\newcommand{\K}{%
  \begin{tikzpicture}[baseline=(A.base),font=\sffamily\small]
    \node[draw=blue, rectangle, rounded corners=2pt,
          text=blue,
          fill=blue!10!white,
          line width=.4pt, inner sep=1pt, outer sep=0pt
          ] (A) {工業的製法};
  \end{tikzpicture}%
}
\newcommand{\R}{%
  \begin{tikzpicture}[baseline=(A.base),font=\sffamily\small]
    \node[draw=red, rectangle, rounded corners=2pt,
          text=red,
          fill=red!10!white,
          line width=.4pt, inner sep=1pt, outer sep=0pt
          ] (A) {例};
  \end{tikzpicture}%
}

%\columnseprule=0.2mm

\pagestyle{fancy}
\lfoot{\hyperlink{top}{無機化学}}
\cfoot{\thepage/\pageref{LastPage}}

\hypersetup{
 bookmarks=false,
 colorlinks=true,
 linkcolor=black,
 citecolor=[rgb]{0,0.4,0.8},
 filecolor=black,
 urlcolor=[rgb]{0,0.4,0.8}
}

\begin{document}

\thispagestyle{fancy}

\begin{center}
\Huge{\gtfamily 無機化学}\\
\vspace{5mm}
\end{center}

\hypertarget{top}{\tableofcontents}
\newpage

 \part{非金属元素}

 \section{水素}
  \hl{無色無臭}の\hl{気体}\footnote{融点 14K 沸点 20K} \hl{最も軽く}、水に溶け\hl{にくい}
  \subsection{同位体}
  \ce{_{}^{1}H} 99\%以上 \ce{_{}^{2}H} (\hl{D})0.015\% \ce{_{}^{3}H} (\hl{T})微量
  \subsection{製法}
  \begin{itemize}
   \item ナフサの電気分解 \K
  \item 赤熱した\hl{コークス}に\hl{水蒸気}を吹き付ける \K\\
  \hce{C + H2O -> H2 + CO}
  \item \hl{水}(\hl{水酸化ナトリウム水溶液})の電気分解\\
  \hce{2H2O -> 2H2 + O2}
  \item \hl{イオン化傾向}が\hl{\ce{H2}より大きい}金属と希薄強酸\\
  \R \;\ce{Fe + 2HCl -> FeCl2 + H2 ^}\\
  \R \;\ce{Zn + 2HCl -> ZnCl2 + H2 ^}
 \end{itemize}
 \subsection{反応}
 \begin{itemize}
  \item 水素と酸素(爆鳴気の燃焼)\\
  \hce{2H2 + O2 -> H2O}
  \item 加熱した酸化銅(\ajRoman{2})と水素\\
  \hce{CuO + H2 -> Cu + H2O}
  \item 水酸化ナトリウムと水\\
  \hce{NaH + H2O -> NaOH + H2}
 \end{itemize}
 \newpage
 \section{貴ガス}
 \hl{\ce{He}}, \hl{\ce{Ne}}, \hl{\ce{Ar}}, \hl{\ce{Kr}}, \ce{Xe}, \ce{Rn}
  \subsection{性質}
  \begin{itemize}
   \item 無色・無臭
   \item 第18族元素であり、電子配置がオクテットを満たすため反応性が低い。
   \item イオン化エネルギーが極めて大きい。
   \item 電子親和力は\hl{極めて小さい}(\hl{ほぼ0})。
   \item 電気陰性度は\hl{定義されない}。
  \end{itemize}
  \subsection{生成}
  \ce{_{}^{40}K}の電子捕獲\\
  \quad\hce{_{}^{40}K + e- -> _{}^{40}Ar}
  \subsection{ヘリウム \ce{He}}
  浮揚ガス
  \subsection{ネオン \ce{Ne}}
  ネオンサイン
  \subsection{アルゴン \ce{Ar}}
  \ce{N2}, \ce{O2}に次いで3番目に空気中での存在量が多い(約1\%)。
  
 \newpage
 \section{ハロゲン}
  \subsection{単体}
  \subsubsection{性質}
  \begin{center}
  \begin{tabular}{|c||c|c|c|c|}\hline
   化学式&\ce{F2}&\ce{Cl2}&\ce{Br2}&\ce{I2}\\ \hline
   分子量&\multicolumn{1}{c}{小}&\multicolumn{2}{c}{\ce{<->}}&\multicolumn{1}{c|}{大}\\ \hline
   分子間力(反応性)&\multicolumn{1}{c}{弱(強)}&\multicolumn{2}{c}{\ce{<->}}&\multicolumn{1}{c|}{強(弱)}\\ \hline
   沸点・融点&\multicolumn{1}{c}{低}&\multicolumn{2}{c}{\ce{<->}}&\multicolumn{1}{c|}{高}\\ \hline
   常温での状態&\hl{気体}&\hl{気体}&\hl{液体}&\hl{固体}\\ \hline
   色&\hl{淡黄}色&\hl{黄緑}色&\hl{赤褐}色&\hl{黒紫}色\\ \hline
   特徴&\hl{特異}臭&\hl{刺激}臭&揮発性&\hl{昇華}性\\ \hline
   \ce{H2}との反応&
   \begin{tabular}{c}
   \hl{冷暗所}でも\\爆発的に反応
   \end{tabular}&
   \begin{tabular}{c}
   \hl{常温}でも\hl{光}で\\爆発的に反応
   \end{tabular}&
   \begin{tabular}{c}
   \hl{加熱}して\\\hl{触媒}により反応
   \end{tabular}&
   \begin{tabular}{c}
   高温で平衡状態\\
   \hl{加熱}して\hl{触媒}により一部反応
   \end{tabular}\\ \hline
   水との反応&
   \begin{tabular}{c}
   水を酸化して酸素を発生\\
   \hl{激しく反応}
   \end{tabular}&
   \hl{一部とけて反応}&
   \hl{一部とけて反応}&
   \begin{tabular}{c}
   \hl{反応しない}\\\hl{\ce{KIaq}には可溶}
   \end{tabular}\\ \hline
   用途&
   \begin{tabular}{c}
   保存が困難\\
   \ce{Kr}や\ce{Xe}と反応
   \end{tabular}&
   \begin{tabular}{c}
   \hl{\ce{ClO-}}による\\\hl{殺菌・漂白}作用
   \end{tabular}&
   \begin{tabular}{c}
   \ce{C=C}や\\
   \ce{C#C}の検出
   \end{tabular}&
   \begin{tabular}{c}
   \hl{ヨウ素デンプン}反応で\\
   \hl{青紫}色
   \end{tabular}\\ \hline
  \end{tabular}
  \end{center}
  \subsubsection{製法}
   \begin{itemize}
    \item フッ化水素ナトリウム\ce{KHF2}のフッ化水素\ce{HF}溶液の電気分解 \K\\
    \ce{KHF2 -> KF + HF}
    \item \hl{水酸化ナトリウム}の電気分解 \K\\
    \hce{2NaCl + 2H2O -> Cl2 + H2 + 2NaOH}
    \item \hl{酸化マンガン(\ajRoman{4})}に\hl{濃硫酸}を加えて加熱\\
    \hce{MnO2 + 4HCl ->[][\Delta] MnCl2 + Cl2 ^ + 2H2O}
    \item \hl{高度さらし粉}と\hl{塩酸}\\
    \hce{Ca(ClO)2*2H2O + 4HCl -> CaCl2 + 2Cl2 ^ + 4H2O}
    \item \hl{さらし粉}と\hl{塩酸}\\
    \hce{CaCl(ClO)*H2O + 2HCl -> CaCl2 + Cl2 ^ + 2H2O}
    \item 臭化マグネシウムと塩素\\
    \ce{MgBr2 + Cl2 -> MgCl2 + Br2}
    \item ヨウ化カリウムと塩素\\
    \ce{2KI + Cl2 -> 2KCl + I2}
   \end{itemize}
  \subsubsection{反応}
  \begin{itemize}
   \item フッ素と水素\\
   \hce{H2 + F2 ->T[常温で爆発的に反応] 2HF}
   \item 塩素と水素\\
   \hce{H2 + Cl2 ->T[光を当てると爆発的に反応] 2HCl}
   \item 臭素と水素\\
   \hce{H2 + Br2 ->T[高温で反応] 2HBr}
   \item ヨウ素と水素\\
   \hce{H2 + I2 <=>T[高温で平衡] 2HI}
   \item フッ素と水\\
   \hce{2F2 + 2H2O -> 4HF + O2}
   \item 塩素と水\\
   \hce{Cl2 + H2O <=> HCl + HClO}
   \item 臭素と水\\
   \hce{Br2 + H2O <=> HBr + HBrO}
   \item ヨウ素の固体がヨウ化物イオン存在下で三ヨウ化物イオンを形成して溶解する反応\\
   \hce{I2 + I- -> I3-}
  \end{itemize}
  
  
  \subsubsection{塩素発生実験の装置}
  \ce{MnO2 + 4HCl ->[][\Delta] MnCl2 + Cl2 ^ + 2H2O}
  \begin{screen}
  \ce{Cl2},\ce{HCl},\ce{H2O}\\
  ↓\hl{\quad 水\quad}に通す (\ce{HCl}の除去)\\
  \ce{Cl2},\ce{H2O}\\
  ↓\hl{濃硫酸}に通す (\ce{H2O}の除去)\\
  \ce{Cl2}
  \end{screen}
  \subsubsection{塩素のオキソ酸}
   \begin{tabular}{r|lrl}
    +\ajRoman{7}&\hl{\ce{HClO4}}&\hl{\quad 過塩素酸}&\hlbox{\chemfig{H-O-Cl(-[:90]O)(-[:-90]O)-O}}\\\hline
    +\ajRoman{5}&\hl{\ce{HClO3}}&\hl{\qquad 塩素酸}&\hlbox{\chemfig{H-O-Cl(-[2]O)-O}}\\\hline
    +\ajRoman{3}&\hl{\ce{HClO2}}&\hl{\quad 亜塩素酸}&\hlbox{\chemfig{H-O-Cl-O}}\\\hline
    +\ajRoman{1}&\hl{\ce{HClO}}&\hl{次亜塩素酸}&\hlbox{\chemfig{H-O-Cl}}\\
   \end{tabular}
  
  \subsection{ハロゲン化水素}
   \subsubsection{性質}
   \begin{center}
   \begin{tabular}{|c||c|c|c|c|}\hline
   化学式&\ce{HF}&\ce{HCl}&\ce{HBr}&\ce{HI}\\ \hline
   色・臭い&\multicolumn{4}{|c|}{\hl{無}色\hl{刺激}臭}\\ \hline
   沸点&$20^\circ$C&$-85^\circ$C&$-67^\circ$C&$-35^\circ$C\\ \hline
   水との反応&\multicolumn{4}{|c|}{\hl{よく溶ける}}\\ \hline
   水溶液&\multicolumn{1}{c}{\hl{フッ化水素酸}}&\multicolumn{1}{c}{\hl{塩酸}}&\multicolumn{1}{c}{\hl{臭化水素酸}}&\hl{ヨウ化水素酸}\\
   (強弱)&
   \multicolumn{4}{|c|}{
   \begin{tabular}{ccccccc}
   \hl{弱酸}&$\ll$&\hl{強酸}&$<$&\hl{強酸}&$<$&\hl{強酸}
   \end{tabular}
   }\\ \hline
   用途&
   \begin{tabular}{c}
   \hl{ガラス}と反応\\
   $\Rightarrow$ ポリエチレン瓶
   \end{tabular}&
   \begin{tabular}{c}
   \hl{アンモニア}の検出\\
   各種工業
   \end{tabular}&
   半導体加工&
   \begin{tabular}{c}
   インジウムスズ\\
   酸化物の加工
   \end{tabular}
   \\ \hline
   \end{tabular}
   \end{center}
   \subsubsection{製法}
    \begin{itemize}
     \item \hl{ホタル石}に\hl{濃硫酸}を加えて加熱(\hl{弱酸遊離})\\
     \hce{CaF2 + H2SO4 ->[][\Delta] CaSO4 + 2HF ^}
     \item \hl{水素}と\hl{塩素} \K\\
     \hce{H2 + Cl2 -> 2HCl ^}
     \item \hl{塩化ナトリウム}に\hl{濃硫酸}に加えて加熱(\hl{弱酸}酸・\hl{揮発性}酸の追い出し)\\
     \hce{NaCl + H2SO4 ->[][\Delta] NaHSO4 + HCl ^}
    \end{itemize}
   \subsubsection{反応}
   \begin{itemize}
    \item 気体のフッ化水素がガラスを侵食する反応\\
    \hce{SiO2 + 4HF(g) -> SiF4 ^ + 2H2O}
    \item フッ化水素酸(水溶液)がガラスを侵食する反応\\
    \hce{SiO2 + 6HF(aq) -> H2SiF6 ^ + 2H2O}
    \item \hl{塩化水素}による\hl{アンモニア}の検出\\
    \hce{AgO2 + 2HF -> 2AgF + H2O}
   \end{itemize}
  \subsection{ハロゲン化銀}
  \subsubsection{性質}
  \begin{tabular}{|c||c|c|c|c|}\hline
  化学式&\ce{AgF}&\ce{AgCl}&\ce{AgBr}&\ce{AgI}\\ \hline
  固体の色&\hl{黄褐}色&\hl{白}色&\hl{淡黄}色&\hl{黄}色\\ \hline
  水との反応&\hl{よく溶ける}&\multicolumn{3}{|c|}{\hl{ほとんど溶けない}}\\ \hline
  光との反応&\hl{感光}&\multicolumn{3}{|c|}{感光性(→\hl{\ce{Ag}})}\\ \hline
  \end{tabular}
  \subsubsection{製法}
  \begin{itemize}
   \item 酸化銀(\ajRoman{1})にフッ化水素酸を加えて蒸発圧縮\\
   \hce{Ag2O + 2HF -> 2AgF + H2O}
   \item ハロゲン化水素イオンを含む水溶液と\hl{硝酸銀水溶液}\\
   \hce{Ag+ + X- -> AgX v}
  \end{itemize}
  \subsection{次亜塩素酸塩}
  \subsubsection{性質}
  \hl{酸化}剤として反応(\hl{殺菌}・\hl{漂白}作用\\
  \quad \hce{ClO- + 2H+ + 2e- -> H2O + Cl-}
  \subsubsection{製法}
  \begin{itemize}
   \item 水酸化ナトリウム水溶液と塩素\\
   \hce{2NaOH + Cl2 -> NaCl + NaClO + H2O}
   \item 水酸化カルシウムと塩素\\
   \hce{Ca(OH)2 + Cl2 -> CaCl(ClO)*H2O}
  \end{itemize}
  \subsection{水素酸カリウム}
  化学式:\hl{\ce{KClO3}}
  \subsubsection{性質}
  \hl{酸素}の生成(\hl{二酸化マンガン}を触媒に加熱)\\
  \quad \hce{2KClO3 ->C[MnO2][\Delta] 2KClO + 3O2 ^}
 \newpage
 \section{酸素}
  \subsection{酸素原子}
  \subsection{酸素}
   \subsubsection{性質}
   地球の地殻に\hl{最も多く}存在
   \begin{itembox}[l]{地球の地殻における元素の存在率}
   \begin{tabular}{ccccccccccc}
   \hl{\ce{O}} & \multirow{2}{*}{$>$} & \hl{\ce{Si}} & \multirow{2}{*}{$>$} & \hl{\ce{Al}} & \multirow{2}{*}{$>$} & \hl{\ce{Fe}} & \multirow{2}{*}{$>$} & \hl{\ce{Ca}} & \multirow{2}{*}{$>$} & \hl{\ce{Na}}\\
   \hl{酸素}&&\hl{ケイ素}&&\hl{アルミニウム}&&\hl{鉄}&&\hl{カルシウム}&&\hl{ナトリウム}\\
   おっ&&し&&ゃる&&て&&か&&な
   \end{tabular}
   \end{itembox}
   \subsubsection{製法}
   \begin{itemize}
    \item \hl{}\K\\
    \item 過酸化水素の分解\\
    \hce{2H2O2 ->C[MnO2] O2 ^ + 2H2O}
    \item 塩素酸カリウムの熱分解\\
    \hce{2KClO3 ->C[MnO2][\Delta] 2KClO + 3O2 ^}
   \end{itemize}
   \subsubsection{反応}
   \hl{酸化}剤としての反応\\
   \quad \hce{O2 + 4H+ + 4e- -> 2H2O}
  \subsection{オゾン}
   化学式:\hl{\ce{O3}}
   \subsubsection{性質}
   \hl{ニンニク}臭(\hl{特異}臭)を持つ\hl{淡青}色の気体(常温)。
   \subsubsection{製法}
   酸素中で\hl{無声放電}/強い\hl{紫外線}を当てる\\
   \quad \hce{3O2 -> 2O3}
   \subsubsection{反応}
   \hl{酸化}剤としての反応\\
   \quad \hce{O3 + 2H+ + 2e- -> O2 + H2O}
 \section{硫黄}
  \subsection{硫黄}
   \subsubsection{性質}
   \begin{tabular}{|c|c|c|c|}\hline
   &\hl{斜方硫黄}&\hl{単斜硫黄}&\hl{ゴム状硫黄}\\ \hline
   色&\hl{}色&\hl{}色&\hl{}色\\ \hline
   融点&$113^\circ$C&$119^\circ$C&不定\\ \hline
   \end{tabular}
  \subsection{硫化水素}
  \subsection{二酸化硫黄(亜硫酸ガス)}
  \subsection{硫酸}
  \subsection{チオ硫酸ナトリウム(ハイポ)}
 \section{窒素}
 \newpage
 \part{金属元素}
\end{document}