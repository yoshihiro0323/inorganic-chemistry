\documentclass[dvipdfmx,jb5]{jsarticle}
\usepackage[top=20truemm,bottom=20truemm,left=20truemm,right=20truemm]{geometry}
\usepackage{multirow}
\usepackage[version=3]{mhchem}
\usepackage{color}
\usepackage{enumerate}
\usepackage{tikz}
\newcommand{\hl}[1]{\underline{\textcolor{red}{#1}}}
\newcommand{\shl}[1]{\underline{#1}}
\pagestyle{empty}

\newcommand{\K}{%
  \begin{tikzpicture}[baseline=(A.base),font=\sffamily\small]
    \node[draw=blue, rectangle, rounded corners=1.5pt,
          text=blue,
          fill=blue!20!white,
          line width=.4pt, inner sep=1pt, outer sep=0pt
          ] (A) {工業的製法};
  \end{tikzpicture}%
}
\newcommand{\R}{%
  \begin{tikzpicture}[baseline=(A.base),font=\sffamily\small]
    \node[draw=red, rectangle, rounded corners=1.5pt,
          text=red,
          fill=red!20!white,
          line width=.4pt, inner sep=1pt, outer sep=0pt
          ] (A) {例};
  \end{tikzpicture}%
}

\begin{document}

\begin{center}
\Large{無機化学}
\end{center}

\part{非金属元素}
 \section{水素}
  \hl{無色無臭}の\hl{気体}\footnote{融点 14K 沸点 20K} \hl{最も軽く}、水に溶け\hl{にくい}
  \subsection{同位体}
  \ce{_{}^{1}H} 99\%以上 \ce{_{}^{2}H} (\hl{D})0.015\% \ce{_{}^{3}H} (\hl{T})微量
  \subsection{製法}
  \begin{itemize}
   \item ナフサの電気分解 \K
  \item 赤熱した\hl{コークス}に\hl{水蒸気}を吹き付ける \K\\
  \shl{\ce{C + H2O -> H2 + CO}}
  \item \hl{水}(\hl{水酸化ナトリウム水溶液})の電気分解\\
  \shl{\ce{2H2O -> 2H + O2}}
  \item \hl{イオン化傾向}が\hl{\ce{H2}より大きい}金属と希薄強酸\\
  \R \;\ce{Zn + 2HCl -> ZnCl2 + H2 ^}
 \end{itemize}
 \subsection{反応}
  \begin{itemize}
   \item 水素と酸素の反応(爆鳴気の燃焼)\\
   \shl{\ce{2H2 + O2 -> H2O}}
   \item フッ素と水素の反応\\
   \shl{\ce{H2 + F2 ->[常温で爆発的に反応] 2HF}}
   \item 塩素と水素の反応\\
   \shl{\ce{Cl2 + F2 ->[光を当てると爆発的に反応] 2HCl}}
   \item ヨウ素と水素の反応\\
   \shl{\ce{H2 + I2 <=>[高音で平衡] 2HI}}
  \end{itemize}
 \section{貴ガス}
 \section{ハロゲン}
\end{document}