\documentclass[dvipdfmx,a4paper,twocolumn]{jsarticle}
\usepackage[top=20truemm,bottom=20truemm,left=20truemm,right=20truemm]{geometry}
\usepackage{multirow}
\usepackage[version=3]{mhchem}
\usepackage{color}
\usepackage{enumerate}
\usepackage{tikz}
\usepackage{lscape}
\usepackage{amsmath}
\usepackage{otf}
\usepackage{chemfig}

\newcommand{\hl}[1]{\underline{\textcolor{red}{\gtfamily #1}}}
\newcommand{\shl}[1]{{\textcolor{blue}{\ce{#1}}}}
\pagestyle{empty}

\newcommand{\K}{%
  \begin{tikzpicture}[baseline=(A.base),font=\sffamily\small]
    \node[draw=blue, rectangle, rounded corners=2pt,
          text=blue,
          fill=blue!10!white,
          line width=.4pt, inner sep=1pt, outer sep=0pt
          ] (A) {工業的製法};
  \end{tikzpicture}%
}
\newcommand{\R}{%
  \begin{tikzpicture}[baseline=(A.base),font=\sffamily\small]
    \node[draw=red, rectangle, rounded corners=2pt,
          text=red,
          fill=red!10!white,
          line width=.4pt, inner sep=1pt, outer sep=0pt
          ] (A) {例};
  \end{tikzpicture}%
}

\columnseprule=0.3mm

\begin{document}

\twocolumn[
\begin{center}
\Huge{\gtfamily 無機化学}\\
\vspace{5mm}
\end{center}
\part{非金属元素}
]
 \section{水素}
  \hl{無色無臭}の\hl{気体}\footnote{融点 14K 沸点 20K} \hl{最も軽く}、水に溶け\hl{にくい}
  \subsection{同位体}
  \ce{_{}^{1}H} 99\%以上 \ce{_{}^{2}H} (\hl{D})0.015\% \ce{_{}^{3}H} (\hl{T})微量
  \subsection{製法}
  \begin{itemize}
   \item ナフサの電気分解 \K
  \item 赤熱した\hl{コークス}に\hl{水蒸気}を吹き付ける \K\\
  \shl{C + H2O -> H2 + CO}
  \item \hl{水}(\hl{水酸化ナトリウム水溶液})の電気分解\\
  \shl{2H2O -> 2H2 + O2}
  \item \hl{イオン化傾向}が\hl{\ce{H2}より大きい}金属と希薄強酸\\
  \R \;\ce{Fe + 2HCl -> FeCl2 + H2 ^}\\
  \R \;\ce{Zn + 2HCl -> ZnCl2 + H2 ^}
 \end{itemize}
 \subsection{反応}
 \begin{itemize}
  \item 水素と酸素(爆鳴気の燃焼)\\
  \shl{2H2 + O2 -> H2O}
  \item 加熱した酸化銅(\ajRoman{2})と水素\\
  \shl{CuO + H2 -> Cu + H2O}
  \item 水酸化ナトリウムと水\\
  \shl{NaH + H2O -> NaOH + H2}
 \end{itemize}
 \newpage
 \section{貴ガス}
 \hl{\ce{He}}, \hl{\ce{Ne}}, \hl{\ce{Ar}}, \hl{\ce{Kr}}, \ce{Xe}, \ce{Rn}
  \subsection{性質}
  \begin{itemize}
   \item 無色・無臭
   \item 第18族元素であり、電子配置がオクテットを満たすため反応性が低い。
   \item イオン化エネルギーが極めて大きい。
   \item 電子親和力は\hl{極めて小さい}(\hl{ほぼ0})。
   \item 電気陰性度は\hl{定義されない}。
  \end{itemize}
  \subsection{生成}
  \ce{_{}^{40}K}の電子捕獲\\
  \quad\shl{_{}^{40}K + e- -> _{}^{40}Ar}
  \subsection{ヘリウム \ce{He}}
  浮揚ガス
  \subsection{ネオン \ce{Ne}}
  ネオンサイン
  \subsection{アルゴン \ce{Ar}}
  \ce{N2}, \ce{O2}に次いで3番目に空気中での存在量が多い(約1\%)。
  
 \newpage
 \section{ハロゲン}
  \subsection{単体}
  \subsubsection{性質}
  \begin{center}
  \rotatebox{90}{
  \begin{tabular}{|c||c|c|c|c|}\hline
   化学式&\ce{F2}&\ce{Cl2}&\ce{Br2}&\ce{I2}\\ \hline
   分子量&\multicolumn{1}{c}{小}&\multicolumn{2}{c}{\ce{<->}}&\multicolumn{1}{c|}{大}\\ \hline
   分子間力(反応性)&\multicolumn{1}{c}{弱(強)}&\multicolumn{2}{c}{\ce{<->}}&\multicolumn{1}{c|}{強(弱)}\\ \hline
   沸点・融点&\multicolumn{1}{c}{低}&\multicolumn{2}{c}{\ce{<->}}&\multicolumn{1}{c|}{高}\\ \hline
   常温での状態&\hl{気体}&\hl{気体}&\hl{液体}&\hl{固体}\\ \hline
   色&\hl{淡黄}色&\hl{黄緑}色&\hl{赤褐}色&\hl{黒紫}色\\ \hline
   特徴&\hl{特異}臭&\hl{刺激}臭&揮発性&\hl{昇華}性\\ \hline
   \ce{H2}との反応&
   \begin{tabular}{c}
   \hl{冷暗所}でも\\爆発的に反応
   \end{tabular}&
   \begin{tabular}{c}
   \hl{常温}でも\hl{光}で\\爆発的に反応
   \end{tabular}&
   \begin{tabular}{c}
   \hl{加熱}して\\\hl{触媒}により反応
   \end{tabular}&
   \begin{tabular}{c}
   高温で平衡状態\\
   \hl{加熱}して\hl{触媒}により一部反応
   \end{tabular}\\ \hline
   水との反応&
   \begin{tabular}{c}
   水を酸化して酸素を発生\\
   \hl{激しく反応}
   \end{tabular}&
   \hl{一部とけて反応}&
   \hl{一部とけて反応}&
   \begin{tabular}{c}
   \hl{反応しない}\\\hl{\ce{KIaq}には可溶}
   \end{tabular}\\ \hline
   用途&
   \begin{tabular}{c}
   保存が困難\\
   \ce{Kr}や\ce{Xe}と反応
   \end{tabular}&
   \begin{tabular}{c}
   \hl{\ce{ClO-}}による\\\hl{殺菌・漂白}作用
   \end{tabular}&
   \begin{tabular}{c}
   \ce{C=C}や\\
   \ce{C#C}の検出
   \end{tabular}&
   \begin{tabular}{c}
   \hl{ヨウ素デンプン}反応で\\
   \hl{青紫}色
   \end{tabular}\\ \hline
  \end{tabular}
  }
  \end{center}
  \subsubsection{製法}
   \begin{itemize}
    \item フッ化水素ナトリウム\ce{KHF2}のフッ化水素\ce{HF}溶液の電気分解 \K
    \item \hl{水酸化ナトリウム}の電気分解 \K\\
    \shl{2NaCl + 2H2O -> Cl2 + H2 + 2NaOH}
    \item \hl{酸化マンガン(\ajRoman{4})}に\hl{濃硫酸}を加えて加熱\\
    \shl{MnO2 + 4HCl ->[][\Delta] MnCl2 + Cl2 ^ + 2H2O}
    \item \hl{高度さらし粉}と\hl{塩酸}\\
    \shl{Ca(ClO)2*2H2O + 4HCl -> CaCl2 + 2Cl2 ^ + 4H2O}
    \item \hl{さらし粉}と\hl{塩酸}\\
    \shl{CaCl(ClO)*H2O + 2HCl -> CaCl2 + Cl2 ^ + 2H2O}
    \item 臭化マグネシウムと塩素\\
    \ce{MgBr2 + Cl2 -> MgCl2 + Br2}
    \item ヨウ化カリウムと塩素\\
    \ce{2KI + Cl2 -> 2KCl + I2}
   \end{itemize}
  \subsubsection{反応}
  \begin{itemize}
   \item フッ素と水素\\
   \shl{H2 + F2 ->T[常温で爆発的に反応] 2HF}
   \item 塩素と水素\\
   \shl{H2 + Cl2 ->T[光を当てると爆発的に反応] 2HCl}
   \item 臭素と水素\\
   \shl{H2 + Br2 ->T[高温で反応] 2HBr}
   \item ヨウ素と水素\\
   \shl{H2 + I2 <=>T[高温で平衡] 2HI}
   \item フッ素と水\\
   \shl{2F2 + 2H2O -> 4HF + O2}
   \item 塩素と水\\
   \shl{Cl2 + H2O <=> HCl + HClO}
   \item 臭素と水\\
   \shl{Br2 + H2O <=> HBr + HBrO}
   \item ヨウ素の固体がヨウ化物イオン存在下で三ヨウ化物イオンを形成して溶解する反応\\
   \shl{I2 + I- -> I3-}
  \end{itemize}
  
  
  \subsubsection{塩素発生実験の装置}
  \ce{MnO2 + 4HCl ->[][\Delta] MnCl2 + Cl2 ^ + 2H2O}
  \ce{Cl2},\ce{HCl},\ce{H2O}\\
  ↓\hl{\quad 水\quad}に通す (\ce{HCl}の除去)\\
  \ce{Cl2},\ce{H2O}\\
  ↓\hl{濃硫酸}に通す (\ce{H2O}の除去)\\
  \ce{Cl2}
  \subsubsection{塩素のオキソ酸}
  \begin{center}
  \rotatebox{90}{
   \begin{tabular}{r|lrl}
    +\ajRoman{7}&\hl{\ce{HClO4}}&\hl{\quad 過塩素酸}&\chemfig{H-O-Cl(-[:90]O)(-[:-90]O)-O}\\\hline
    +\ajRoman{5}&\hl{\ce{HClO3}}&\hl{\qquad 塩素酸}&\chemfig{H-O-Cl(-[2]O)-O}\\\hline
    +\ajRoman{3}&\hl{\ce{HClO2}}&\hl{\quad 亜塩素酸}&\chemfig{H-O-Cl-O}\\\hline
    +\ajRoman{1}&\hl{\ce{HClO}}&\hl{次亜塩素酸}&\chemfig{H-O-Cl}\\
   \end{tabular}
  }
  \end{center}
  
  \subsection{ハロゲン化水素}
   \subsubsection{性質}
   \begin{center}
   \rotatebox{90}{
   \begin{tabular}{|c||c|c|c|c|}\hline
   化学式&\ce{HF}&\ce{HCl}&\ce{HBr}&\ce{HI}\\ \hline
   色・臭い&\multicolumn{4}{|c|}{\hl{無}色\hl{刺激}臭}\\ \hline
   沸点&$20^\circ$C&$-85^\circ$C&$-67^\circ$C&$-35^\circ$C\\ \hline
   水との反応&\multicolumn{4}{|c|}{\hl{よく溶ける}}\\ \hline
   水溶液&\multicolumn{1}{c}{\hl{フッ化水素酸}}&\multicolumn{1}{c}{\hl{塩酸}}&\multicolumn{1}{c}{\hl{臭化水素酸}}&\hl{ヨウ化水素酸}\\
   (強弱)&
   \multicolumn{4}{|c|}{
   \begin{tabular}{ccccccc}
   \hl{弱酸}&$\ll$&\hl{強酸}&$<$&\hl{強酸}&$<$&\hl{強酸}
   \end{tabular}
   }\\ \hline
   用途&
   \begin{tabular}{c}
   \hl{}と反応\\
   ポリエチレン板
   \end{tabular}&
   \begin{tabular}{c}
   \hl{}の検出\\
   各種工業
   \end{tabular}&
   半導体加工&
   \begin{tabular}{c}
   インジウムスズ\\
   酸化物の加工
   \end{tabular}
   \\ \hline
   \end{tabular}
   }
   \end{center}
   \subsubsection{製法}
    \begin{itemize}
     \item \hl{ホタル石}に\hl{濃硫酸}を加えて加熱\\
     \shl{CaF2 + H2SO4 ->[][\Delta] CaSO4 + 2HF ^}
     \item \hl{水素}と\hl{塩素} \K\\
     \shl{H2 + Cl2 -> 2HCl}
     \item \hl{塩化ナトリウム}に\hl{濃硫酸}に加えて加熱(揮発性酸の追い出し)\\
     \shl{NaCl + H2SO4 ->[][\Delta] NaHSO4 + HCl ^}
    \end{itemize}
  \subsection{ハロゲン化銀}
  \subsubsection{性質}
  \begin{center}
  \rotatebox{0}{
  \begin{tabular}{|c||c|c|c|c|}\hline
  化学式&\ce{AgF}&\ce{AgCl}&\ce{AgBr}&\ce{AgI}\\ \hline
  固体の色&\hl{}色&\hl{}色&\hl{}色&\hl{}色\\ \hline
  水との反応& &\multicolumn{3}{|c|}{}\\ \hline
  光&\multicolumn{4}{|c|}{}\\ \hline
  \end{tabular}
  }
  \end{center}
 \newpage
 \twocolumn[
 \part{金属元素}
 ]
\end{document}