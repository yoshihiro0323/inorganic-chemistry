%典型元素
\part{典型金属}
 \section{アルカリ金属}
 \subsection{単体}
 \subsubsection{性質}
 \begin{itemize}
  \item 銀白色で\hl{柔らかい}金属
  \item 全体的に反応性が高く、\hl{灯油}中に保存
  \item 原子一個あたりの自由電子が\hl{1}個(\hl{弱}い\hl{金属}結合)
  \item 還元剤として反応\\
  \hce{M -> M+ + e-}
 \end{itemize}
 \begin{tabular}{|c||c|c|c|c|c|}\hline
 化学式&\ce{Li}&\ce{Na}&\ce{K}&\ce{Rb}&\ce{Cs}\\ \hline
 融点&$181^{\circ}$C& $98^{\circ}$C& $64^{\circ}$C& $39^{\circ}$C& $28^{\circ}$C\\\hline
 密度&0.53&0.97&0.86&1.53&1.87\\ \hline
 構造&\multicolumn{5}{|c|}{\hl{体心立方}格子(\hl{軽金属})}\\ \hline
 イオン化エネルギー&\multicolumn{5}{|c|}{大 \quad\tikz \draw[line width=0.5pt] (0,0.1)--(10,0)--(0,-0.1);\quad 小}\\ \hline
 反応力&\multicolumn{5}{|c|}{小 \quad\tikz \draw[line width=0.5pt] (10,0.1)--(0,0)--(10,-0.1);\quad 大}\\ \hline
 炎色反応&\hl{赤}色&\hl{黄}色&\hl{赤紫}色&\hl{深赤}色&\hl{青紫}色\\ \hline
 用途&
 \begin{tabular}{l}
 リチウムイオン\\
 電池の負極
 \end{tabular}&
 \begin{tabular}{l}
 トンネル照明\\
 高速増殖炉の冷却材
 \end{tabular}&
 \begin{tabular}{l}
 磁気センサー\\
 肥料(\ce{K+})
 \end{tabular}&
 \begin{tabular}{l}
 光電池\\
 年代測定
 \end{tabular}&
 \begin{tabular}{l}
 光電管\\
 電子時計\\
 (一秒の基準)
 \end{tabular}\\\hline
 \end{tabular}
 \subsubsection{製法}
 水酸化物や塩化物の\hl{溶融塩電解}(\hl{ダウンズ}法) \K\\
 \hl{\ce{CaCl2}}添加(\hl{凝固点降下})\\
 \hce{2NaCl -> 2Na + Cl2 ^}
 \subsubsection{反応}
 \begin{itemize}
  \item ナトリウムと酸素\\
  \hce{4Na + O2 -> 2Na2O}
  \item ナトリウムと塩素\\
  \hce{2Na + Cl2 -> 2NaCl}
  \item ナトリウムと水\\
  \hce{2Na + 2H2O -> 2NaOH + H2 ^}
 \end{itemize}
 \subsection{水酸化ナトリウム(苛性ソーダ)}
 化学式:\hl{\ce{NaOH}}
 \subsubsection{性質}
  \begin{itemize}
   \item \hl{白}色の固体
   \item \hl{潮解}性
   \item 水によくとける(水との親和性が\hl{非常に高い})
   \item \hl{乾燥}剤
   \item 強塩基性\\
   $\left(
   \begin{tabular}{ll}
   \hl{\ce{NaOH <=> Na+ + OH-}}&$K_{1} = 1.0\times10^{-1}$mol/L
   \end{tabular}
   \right)$
   \item 空気中の\hl{二酸化炭素}と反応して、純度が不明\\
   酸の標準溶液(\hl{シュウ酸})を用いた中和滴定で濃度決定\\
   $\left(
   \begin{tabular}{l}
   \hce{(COOH)2 + 2 NaOH -> (COONa)2 + 2H2O}
   \end{tabular}
   \right)$
  \end{itemize}
 \subsubsection{製法}
 \hl{水酸化ナトリウム水溶液}の\hl{電気分解}(イオン交換膜法) \K\\
 \hce{2NaCl + 2H2O -> 2NaOH + H2 ^ + Cl2 ^}
 \subsubsection{反応}
 \begin{itemize}
  \item 塩酸と水酸化ナトリウム\\
  \hce{HCl + NaOH -> NaCl + H2O}
  \item 塩素と水酸化ナトリウム\\
  \hce{2NaOH + Cl2 -> NaCl + NaClO + H2O}
  \item 二酸化硫黄と水酸化ナトリウム\\
  \hce{SO2 + 2NaOH -> Na2SO3 + H2O}
  \item 酸化亜鉛と水酸化ナトリウム水溶液\\
  \hce{ZnO + 2NaOH + H2O -> Na2[Zn(OH)4]}
  \item 二酸化炭素と水酸化ナトリウム\\
  \hce{2NaOH + CO2 -> Na2CO3 + H2O}
 \end{itemize}
 \subsection{炭酸ナトリウム・炭酸水素ナトリウム}
 \subsubsection{性質}
 \begin{tabular}{|c||c|c|}\hline
 名称&炭酸ナトリウム&炭酸水素ナトリウム\\ \hline
 化学式&\hl{\ce{Na2CO3}}&\hl{\ce{NaHCO3}}\\ \hline
 色&\hl{白}色&\hl{白}色\\ \hline
 融点&$850^{\circ}$C&\hl{熱分解}\\ \hline
 液性&\hl{塩基}性&\hl{弱塩基}性\\ \hline
 用途&\hl{ガラス}や石鹸の原料&胃腸薬・ふくらし粉\\ \hline
 \end{tabular}
 \subsubsection{製法}
\begin{itembox}[l]{\hl{アンモニアソーダ法}(\hl{ソルベー法})\K}
\begin{tikzpicture}[box/.style={draw=black, rectangle, rounded corners=4pt, line width=1pt, inner sep=5pt, outer sep=0pt},arrow/.style={-latex,very thick}]
 \node[box, right, fill=orange!20!white] (CaCO3) {\hl{\ce{CaCO3}}};
 \node[box, right] (CaO) at(3,0){\hl{\ce{CaO}}};
 \node[box, right] (CaOH2) at(6,0){\hl{\ce{Ca(OH)2}}};
 \node[box, right, fill=blue!10!white] (CaCl2) at(11,0){\hl{\ce{CaCl2}}};
 \draw[arrow, red] (CaCO3.east)--node[auto=left] {1}(CaO.west);
 \draw[arrow, blue] (CaO.east)--node[auto=left] {2}(CaOH2.west);
 \draw[arrow] (CaOH2.east)-- node[auto=left] {5}(CaCl2.west);
 
 \coordinate (1) at ($(CaCO3.east)!0.5!(CaO.west)$);
 \node[box] (CO2) at ([xshift=-5mm,yshift=-15mm]1){\hl{\ce{CO2}}};
 
 \coordinate (2) at ($(CaO.east)!0.5!(CaOH2.west)$);
 \node[box] (H2O) at ([yshift=-15mm]2){\hl{\ce{H2O}}};
 \draw[arrow, red] (1)--($(CO2.north)!(1)!(CO2.north east)$);
 \draw[arrow, blue] (H2O.north)--(2);
 \node[box, right, fill=green!10!white] (NaCl) at(0,-3) {\hl{\ce{NaCl}}};
 \node[box, right] (NaHCO3) at(2.5,-3){\hl{\ce{NaHCO3}}};
 \node[box, right] (Na2CO3) at(6,-3){\hl{\ce{Na2CO3}}};
 \draw[arrow, green!75!black] (NaCl.east)--node[auto=right] {3}(NaHCO3.west);
 \draw[arrow, orange] (NaHCO3.east)--node[auto=right] {4}(Na2CO3.west);
 
 \coordinate (3-1) at ($(NaCl.east)!0.25!(NaHCO3.west)$);
 \coordinate (3-3) at ($(NaCl.east)!0.75!(NaHCO3.west)$);
 \coordinate (4-2) at ($(NaHCO3.east)!0.25!(Na2CO3.west)$);
 \draw[arrow, green!75!black] (CO2.south)--($(NaCl.east)!(CO2.south)!(NaHCO3.west)$);
 \node[box] (NH3) at ([yshift=-15mm]3-1){\hl{\ce{NH3}}};
 \draw[arrow, green!75!black] (NH3.north)--(3-1.south);
 \node[box] (NH4Cl) at ([yshift=-15mm]4-2){\hl{\ce{NH4Cl}}};
 \draw[arrow, green!75!black] (3-3)--($(3-3)+(0,-0.5)$)-|(NH4Cl.north);
 \draw[arrow, orange] (4-2)|-($(CO2.south)!.5!(CO2.south east)+(0,-0.5)$)--($(CO2.south)!.5!(CO2.south east)$);
 \draw[arrow, orange] ($(NaHCO3.east)!(H2O.south)!(Na2CO3.west)$)--(H2O.south);
 \coordinate (5-1) at ($(CaOH2.east)!0.25!(CaCl2.west)$);
 \coordinate (5-2) at ($(CaOH2.east)!0.5!(CaCl2.west)$);
 \coordinate (5-3) at ($(CaOH2.east)!0.75!(CaCl2.west)$);
 \draw[arrow] (NH4Cl.east)-|(5-1);
 \draw[arrow] (5-2)|-(H2O.east);
 \draw[arrow] (5-3)|-($(NH3.south)+(0,-0.5)$)--(NH3.south);
\end{tikzpicture}
\begin{enumerate}
 \item \hl{炭酸カルシウム}の\hl{熱分解}\\
 \hce{CaCO3 -> CaO + CO2}
 \item \hl{酸化カルシウム}と\hl{水}\\
 \hce{CaO + H2O -> Ca(OH)2}
 \item \hl{塩化ナトリウム水溶液}に\hl{アンモニア}を溶解させてから、\hl{二酸化炭素}を溶解\\
 \hce{NaCl + NH3 + CO2 + H2O -> NH4Cl + NaHCO3 v}
 \item \hl{炭酸水素ナトリウム}の\hl{熱分解}\\
 \hce{2NaHCO3 -> Na2CO3 + H2O + CO2 ^}
 \item \hl{水酸化カルシウム}と\hl{塩化アンモニウム}\\
 \hce{2NH4Cl + Ca(OH)2 -> CaCl2 + 2NH3 ^ + 2H2O}
\end{enumerate}
\hce{2NaCl + CaCO3 -> CaCl2 + Na2CO3}
\end{itembox}

 \subsubsection{反応}
 \begin{itemize}
  \item \ce{Na2CO3}
  \begin{tabular}{ll}
  \hl{\ce{CO3^2- + H2O <=> HCO3- + OH-}}&$K_{1}=1.8\times10^{-4}$\\
  \end{tabular}
  \item \ce{NaHCO3}
  $\left\{
  \begin{tabular}{ll}
  \hl{\ce{HCO3- <=> H+ + CO3^{2-}}}&$K_{1}=5.6\times10^{-11}$\\
  \hl{\ce{HCO3- + H2O <=> CO2 + OH- + H2O}}&$K_{2}=2.3\times10^{-8}$
  \end{tabular}
  \right.$
 \end{itemize}
 
\newpage
 \section{2族元素}
 \hl{\ce{Be}},\hl{\ce{Mg}},\hl{アルカリ土類金属}
 \subsection{単体}
 \subsubsection{性質}
 \begin{tabular}{|c||c|c|c|c|c|}\hline
 化学式&\hl{\ce{Be}}&\hl{\ce{Mg}}&\hl{\ce{Ca}}&\hl{\ce{Sr}}&\hl{\ce{Ba}}\\ \hline
 融点&$1282^{\circ}$C&$649^{\circ}$C&$839^{\circ}$C&$769^{\circ}$C&$729^{\circ}$C\\ \hline
 密度(g/cm$^3$)&1.85&1.74&1.55&2.54&3.59\\ \hline
 \hl{還元}力&\multicolumn{5}{|c|}{小 \quad\tikz \draw[line width=0.5pt] (6,0.1)--(0,0)--(6,-0.1);\quad 大}\\ \hline
 水との反応&\hl{反応しない}&\hl{熱水}&\hl{冷水}&\hl{冷水}&\hl{冷水}\\ \hline
 \ce{M(OH)2}の水溶性&\multicolumn{2}{|c|}{\hl{難溶}性(\hl{弱塩基}性)}&\multicolumn{3}{|c|}{\hl{可溶}性(\hl{強塩基}性)}\\ \hline
 難溶性の塩&\multicolumn{2}{|c|}{\hl{\ce{MCO3}}}&\multicolumn{3}{|c|}{\hl{\ce{MCO3,MSO4}}}\\ \hline
 炎色反応&\hl{示さない}&\hl{示さない}&\hl{橙赤}&\hl{紅}&\hl{黄緑}\\ \hline
 用途&X線通過窓&フラッシュ&精錬の還元剤&発煙筒&ゲッター\\ \hline
 \end{tabular}
 \subsubsection{製法}
 塩化物の\hl{溶融塩電解} \K
 \subsubsection{反応}
 \begin{itemize}
  \item マグネシウムの燃焼\\
  \hce{2Mg + O2 -> 2 MgO}
  \item マグネシウムと二酸化炭素\\
  \hce{2Mg + CO2 -> 2 MgO + C}
  \item カルシウムと水\\
  \hce{Ca + 2H2O -> Ca(OH)2 + H2 ^}
 \end{itemize}
 
 \subsection{酸化カルシウム(生石灰)}
 化学式:\hl{\ce{CaO}}
 \subsubsection{性質}
 \begin{itemize}
  \item \hl{白}色
  \item \hl{水}との親和性が\hl{非常に高い}(\hl{乾燥剤})
  \item \hl{塩基性}酸化物
  \item 水との反応熱が\hl{非常に大きい}(\hl{加熱剤})
 \end{itemize}
 \subsubsection{製法}
 \hl{炭酸カルシウム}の\hl{熱分解}\\
 \hce{CaCO3 -> CaO + CO2}
 \subsubsection{反応}
 \begin{itemize}
  \item コークスを混ぜて強熱すると、\hl{炭化カルシウム}(\hl{カーバイド})が生成\\
  \hce{CaO + 3C -> CaC2 + CO ^}\\
  \hl{水}と反応して\hl{アセチレン}が生成\\
  \hce{CaC2 + 2H2O -> C2H2 ^ + Ca(OH)2}
 \end{itemize}
 \subsection{水酸化カルシウム(消石灰)}
 化学式:\hl{\ce{Ca(OH)2}}
 \subsubsection{性質}
 \begin{itemize}
  \item \hl{白}色
  \item 水に\hl{少し溶ける}固体
  \item \hl{強塩基}
  $\left(
  \begin{tabular}{ll}
  \hl{\ce{Ca(OH)2 <=> Ca(OH)+ + OH-}}&$K_{1}=5.0\times10^{-2}$\\
  \end{tabular}
  \right)$
  \item 水溶液は\hl{石灰水}
 \end{itemize}
 \subsubsection{製法}
 \hl{酸化カルシウム}と\hl{水} \K\\
 \hce{CaO + H2O -> Ca(OH)2}
 \subsubsection{反応}
 \begin{itemize}
  \item 塩素と反応して、\hl{さらし粉}が生成\\
  \hce{Ca(OH)2 + Cl2 -> CaCl(ClO)*H2O}
  \item $580^{\circ}$C以上で\hl{熱分解}\\
  \hce{Ca(OH)2 -> CaO + H2O}
  \item 二酸化炭素との反応\\
  \hce{Ca(OH)2 + CO2 -> CaCO3 + H2O}
  \item 塩化アンモニウムとの反応\\
  \hce{2NH4Cl + Ca(OH)2 -> CaCl2 + 2NH3 ^ + 2H2O}
 \end{itemize}
 \subsection{炭酸カルシウム(石灰石)}
 化学式:\hl{\ce{CaCO3}}
 \subsubsection{性質}
 \begin{itemize}
  \item \hl{白}色で、水に\hl{溶けにくい}
  \item \hl{鍾乳洞}の形成
 \end{itemize}
 \subsubsection{反応}
 \begin{itemize}
  \item $800^{\circ}$C以上で\hl{熱分解}\\
  \hce{CaCO3 -> CaO + CO2}
  \item \hl{二酸化炭素}を多く含む水に\hl{溶解}\\
  \hce{CaCO3 + CO2 + H2O <=> Ca(HCO3)2}
 \end{itemize}
 \subsection{塩化マグネシウム・塩化カルシウム}
 化学式:\hl{\ce{MgCl2}}・\hl{\ce{CaCl2}}
 \subsubsection{性質}
  \hl{潮解}性があり、水に\hl{よく溶ける}(水との親和性が\hl{非常に高い})\\
  \hl{乾燥}剤 \stamp{teal}{塩化カルシウム}、\hl{融雪}剤
 \subsubsection{製法}
 \begin{itemize}
  \item 海水から得た\hl{にがり}を濃縮 \stamp{teal}{塩化マグネシウム} \K
  \item \hl{アンモニアソーダ法}(\hl{ソルベー法})\stamp{teal}{塩化カルシウム} \K
 \end{itemize}
 \subsection{硫酸カルシウム}
 化学式:\hl{\ce{CaSO4}}
 \subsubsection{性質}
  \hl{セッコウ}を約$150^{\circ}$Cで加熱すると、\hl{焼きセッコウ}が生成\\
  \hl{水}を加えると、\hl{発熱}・\hl{膨張}・\hl{硬化}して\hl{セッコウ}に戻る\\
  \hce{CaSO4*2H2O <=>T[$\Delta$][硬化] CaSO4*$\dfrac{1}{2}$H2O + $\dfrac{3}{2}$H2O}\\
  \stamp{red}{用途} 医療用ギプス・石膏像・建材
 \subsection{硫酸バリウム}
 化学式:\hl{\ce{BaSO4}}
 \subsubsection{性質}
 \begin{itemize}
  \item \hl{白}色で、水に\hl{ほとんど溶けない}固体
  \item 反応性が\hl{低}く、X線を遮蔽
 \end{itemize}
 \section{12族元素}
 \subsection{単体}
 \subsubsection{性質}
 \begin{tabular}{|c|c|c|c|}\hline
 化学式&\hl{\ce{Zn}}&\hl{\ce{Cd}}&\hl{\ce{Hg}}\\ \hline
 融点&$420^\circ$C&$321^\circ$C&$-39^\circ$C\\ \hline
 密度&7.1&8.6&13.6\\ \hline
 \ce{M^{2+}aq + H2S}&\hl{白}色の\hl{\ce{ZnS}}\ce{v}&\hl{黄}色の\hl{\ce{CdS}}\ce{v}&\hl{黒}色の\hl{\ce{HgS}}\ce{v}\\
 (沈澱条件)&(\hl{中塩基性})&(\hl{全液性})&(\hl{全液性})\\ \hline
 \multirow{2}{*}{特性}&高温の水蒸気と反応&\ce{Cd^2+}は\ce{Ca^2+}と類似&\hl{合金}を作りやすい\\
 &\hl{両性}元素&$\Rightarrow$イタイイタイ病&(\hl{アマルガム})\\ \hline
 用途&\hl{トタン}(鉄にメッキ)&ニカド電池 (Ni-Cd)&体温計・蛍光灯\\ \hline
 \end{tabular}
 \begin{itemize}
  \item 12族の硫化物は\hl{顔料}や\hl{染料}に利用
  \item \ce{HgS}は$450^\circ$Cで消火させると\hl{赤}色に変化
 \end{itemize}
 \subsubsection{製法}
 閃亜鉛鉱を焙焼して得た酸化亜鉛に、コークスを混ぜて加工 \K\\
 \hce{2ZnS + 3O2 -> 2ZnO + 2SO2}\\
 \hce{ZnO + C -> Zn + CO}
 \subsubsection{反応}
 \begin{itemize}
  \item 高温の水蒸気と反応 \stamp{teal}{亜鉛}\\
  \hce{Zn + H2O -> ZnO + H2 ^}
  \item 塩酸と反応 \stamp{teal}{亜鉛}\\
  \hce{Zn + 2HCl -> ZnCl2 + H2 ^}
  \item 水酸化ナトリウム水溶液と反応 \stamp{teal}{亜鉛}\\
  \hce{Zn + 2NaOH + 2H2O -> Na2[Zn(OH)4] + H2 ^}
 \end{itemize}
 \subsection{酸化亜鉛(亜鉛華)・水酸化亜鉛}
 化学式:\hl{\ce{ZnO}}・\hl{\ce{Zn(OH)2}}
 \subsubsection{性質}
 \begin{itemize}
  \item \hl{白}色で、水に\hl{とけにくい}固体
  \item 酸化亜鉛は\hl{顔料}
  \item \hl{両性}酸化物/水酸化物\\
  \hl{酸}・(強)\hl{塩基}と反応
  \ce{Zn^2+}は、\hl{\ce{OH-}}とも\hl{\ce{NH3}}とも錯イオンを形成
 \end{itemize}
 \subsubsection{製法}
 \begin{itemize}
  \item 亜鉛を燃焼 \K \stamp{teal}{酸化亜鉛}\\
  \hce{2Zn + O2 -> 2ZnO}
  \item 亜鉛イオンを含む水溶液に、少量の\hl{\ce{OH-}}を加える \stamp{teal}{水酸化亜鉛}\\
  \hce{Zn^2+ + 2OH- -> Zn(OH)2 v}
 \end{itemize}
 \subsubsection{反応}
 \begin{itemize}
  \item 酸化亜鉛と塩酸\\
  \hce{ZnO + 2HCl -> ZnCl2 + H2O}
  \item 酸化亜鉛と水酸化ナトリウム水溶液\\
  \hce{ZnO + 2NaOH + H2O -> Na2[Zn(OH)4]}
  \item 水酸化亜鉛と塩酸\\
  \hce{Zn(OH)2 + 2HCl -> ZnCl2 + 2H2O}
  \item 水酸化亜鉛と水酸化ナトリウム水溶液\\
  \hce{Zn(OH)2 + 2NaOH -> Na2[Zn(OH)4]}
  \item 水酸化亜鉛の過剰な\hl{アンモニア}との反応\\
  \hce{Zn(OH)2 + 4NH3 -> [Zn(NH3)4](OH)2}
 \end{itemize}
 \subsection{塩化水銀(\UTF{2160})・塩化水銀(\UTF{2161})}
 化学式:\hl{\ce{Hg2Cl2}}・\hl{\ce{HgCl}}
 \subsubsection{性質}
 \begin{itemize}
  \item 白色で、水に溶けにくい固体で、微毒 \stamp{teal}{塩化水銀(\UTF{2160})}
  \item 白色で、水に少し溶ける固体で、猛毒 \stamp{teal}{塩化水銀(\UTF{2161})}
 \end{itemize}
 \subsubsection{製法}
 水酸化銀(\ajRoman{2})と水銀の混合物を加熱\\
 \hce{HgCl2 + Hg -> Hg2Cl2}
 
 \section{アルミニウム}
 \subsection{アルミニウム}
 \subsubsection{性質}
 \begin{itemize}
  \item 密度が\hl{小さく}、\hl{やわからかい}金属
  \item 展性・延性が\hl{大きく}、電気・熱伝導率が\hl{高い}
  \begin{itembox}[l]{電気・熱伝導性が高い金属}
  \hl{\ce{Ag}}>\hl{\ce{Cu}}>\hl{\ce{Au}}>\hl{\ce{Al}}
  \end{itembox}
  \item \hl{両性}元素(\hl{濃硝酸}には\hl{不動態}となり反応しない)\\
  表面の緻密な\hl{酸化被膜}が内部を保護(\hl{\ce{Al}},\hl{\ce{Cr}},\hl{\ce{Fe}},\hl{\ce{Co}},\hl{\ce{Ni}}\footnote{てつこに})\\
  電気分解(\hl{陽}極)で人工的に厚い酸化被膜をつける製品加工(\hl{アルマイト})
  \item イオン化傾向が\hl{大きく}、\hl{還元}力が\hl{高い}
  \item \hl{テルミット}反応(多量の\hl{熱}・\hl{光}が発生)
 \end{itemize}
 \subsubsection{製法}
 \begin{itemize}
  \item \hl{ボーキサイト}から得た\hl{酸化アルミニウム}(\hl{アルミナ})の溶融塩電解 \K
  \item バイヤー法
  \begin{enumerate}
   \item \hl{ボーキサイト}を濃い\hl{水酸化ナトリウム}水溶液に溶解\\
   \hce{Al2O3 + 2NaOH + 3H2O -> 2Na[Al(OH)4]}
   \item 溶解しない不純物をろ過して、ろ液を水で希釈してAl(OH)3の種結晶を入れる\\
   \hce{Na[Al(OH)4] -> NaOH + Al(OH)3 v} 
   \item 成長した\hl{\ce{Al(OH)3}}を強熱\\
   \hce{2Al(OH)3 -> Al2O3 + 3H2O}
  \end{enumerate}
  \item ホールエール法
  \begin{enumerate}
   \item \hl{氷晶石}\ce{Na3AlF6}を融解し、酸化アルミニウムを溶解
   \item \hl{炭素}電極で電気分解
   $\left\{
   \begin{tabular}{ll}
   陽極&\hce{C + O^2- -> CO + 2 e-},\hce{C + 2O^2- -> CO2 + 4e-}\\
   陰極&\hce{Al3+ + 3e- -> Al}
   \end{tabular}
   \right.$
  \end{enumerate}
 \end{itemize}
 \subsubsection{反応}
 \begin{enumerate}
  \item アルミニウムの燃焼\\
  \hce{4Al + 3O2 -> 2Al2O3}
  \item アルミニウムと高温の水蒸気\\
  \hce{2Al + 3H2O -> Al2O3 + 3H2 ^}
  \item テルミット反応\\
  \hce{Fe2O3 + 2Al -> Al2O3 + 2Fe}
 \end{enumerate}
 \newpage
 \subsection{酸化アルミニウム・水酸化アルミニウム}
 化学式:\hl{\ce{Al2O3}}・\hl{\ce{Al(OH)3}}
 酸化アルミニウムの別称:\hl{アルミナ}
 \subsubsection{性質}
 \begin{itemize}
  \item \hl{白}色で、水に\hl{溶けにくい}\\
  \item \hl{両性}酸化物/水酸化物\\
  \hl{酸}・(強)\hl{塩基}と反応\\
  \ce{Al^3+}は\hl{\ce{OH-}}と錯イオンを形成し、\hl{\ce{NH3}}とは形成しない
 \end{itemize}
 \subsubsection{製法}
 \begin{itemize}
  \item バイヤー法
  \item アルミニウムイオンを含む水溶液に、少量の\hl{塩基}を加える \stamp{teal}{水酸化アルミニウム}\\
  \hce{Al3+ + 3OH- -> Al(OH)3 v}
 \end{itemize}
 \subsubsection{反応}
 \begin{itemize}
 \item 酸化アルミニウムと塩酸\\
  \hce{Al2O3 + 6HCl -> 2AlCl3 + 3H2O}
  \item 酸化アルミニウムと水酸化ナトリウム水溶液\\
  \hce{Al2O3 + 2NaOH + 3H2O -> 2Na[Al(OH)4]}
  \item 水酸化アルミニウムと塩酸\\
  \hce{Al(OH)3 + 3HCl -> AlCl3 + 3H2O}
  \item 水酸化アルミニウムと水酸化ナトリウム水溶液\\
  \hce{Al(OH)3 + NaOH -> Na[Al(OH)4]}
 \end{itemize}
 \subsection{ミョウバン・焼きミョウバン}
 化学式:\hl{\ce{AlK(SO4)2*12H2O}}・\hl{\ce{AlK(SO4)2}}
 \subsubsection{性質}
 \begin{itemize}
  \item \hl{白}色で、水に\hl{溶ける}固体
  \item \hl{酸性}\\
   $\left(
   \begin{tabular}{ll}
    \hl{\ce{Al^3+ + H2O <=> Al(OH)2 + H+}}&$K_{1}=1.1\times10^{-5}$ mol/L
   \end{tabular}
   \right)$
  \item \ce{Al^3+}は価数が\hl{大きい}陽イオン\\
  粘土(\hl{負}の\hl{疎水}コロイド)で濁った水の浄水処理(\hl{凝析})
  \item 水への溶解\\
  \hce{AlK(SO4)2 -> Al3+ + K+ + SO4^2-}
 \end{itemize}
 \subsubsection{製法}
 硫酸化アルミニウムと硫酸カリウムの混合水溶液を濃縮
 \section{スズ・鉛}
 \subsection{単体}
 \subsubsection{性質}
 \begin{tabular}{|c|c|c|}\hline
 化学式&\hl{\ce{Sn}}&\hl{\ce{Pb}}\\ \hline
 特徴&灰白色で柔らかい金属&青白色で柔らかい金属\\ \hline
 融点&$232^\circ$C&$328^\circ$C\\ \hline
 密度&7.28&11.4\\ \hline
 特性&\multicolumn{2}{|c|}{\hl{両性}元素}\\ \hline
 \multirow{2}{*}{用途}&\hl{ブリキ}(鉄にメッキ)&\hl{鉛蓄}電池の\hl{負}極\\
 &\multicolumn{2}{|c|}{\hl{放射線}の遮蔽}\\ \hline
 \end{tabular}\\
 【合金】\\
 \ce{Cu + Sn}$\cdots$\hl{青銅}\\
 \ce{Sn + Pb}$\cdots$\hl{はんだ}
 \subsubsection{製法}
 \begin{itemize}
  \item 錫石\ce{SnO2}にコークスを混ぜて加熱 \K  \stamp{teal}{スズ}\\
  \hce{SnO2 + 2C -> Sn + 2CO}
  \item 方鉛鉱\ce{PbS}を焙焼してから、コークスを混ぜて加熱 \K  \stamp{teal}{鉛}\\
  \hce{2PbS + 3O2 -> 2PbO + 2 SO2}\\
  \hce{PbO + C -> Pb + CO}
 \end{itemize}
 \subsubsection{反応}
 \begin{itemize}
  \item 鉛と\hl{希硝}酸\\
  \hce{3Pb + 8HNO3 -> 3Pb(NO3)2 + 4H2O + 2NO}
  \item 鉛と\hl{酢}酸\\
  \hce{2Pb + 4CH3COOH + O2 -> 2(CH3COO)2Pb + 2H2O}
  \item スズと\hl{塩酸}\\
  \hce{Sn + 2HCl -> SnCl2 + H2 ^}
  \item 鉛蓄電池における反応\\
  \hce{Pb + PbO2 + 2H2SO4 <=>T[放電][充電] 2PbSO4 + 2H2O}
  $\left\{
  \begin{tabular}{ll}
  正極&\hce{PbO2 + SO4^2- + 4H+ + 2e- -> PbSO4 + 2H2O}\\
  負極&\hce{Pb + SO4^2- -> PbSO4 + 2e-}
  \end{tabular}
  \right.$
 \end{itemize}
 \subsection{塩化スズ(\UTF{2161})}
 \subsubsection{性質}
 \hl{還元}剤として働く\\
 \hce{PbO2 + 4H+ + 2e- -> Pb^{2+} + 2H2O}
 \subsubsection{製法}
 スズと\hl{塩酸}\\
 \hce{Sn + 2HCl -> SnCl2 + H2 ^}
 \subsubsection{反応}
 塩化鉄(\ajRoman{3})水溶液と塩化スズ(\ajRoman{2})水溶液\\
 \hce{2FeCl3 + SnCl2 -> 2FeCl2 + SnCl4}\\
 \stamp{black}{備考} 塩化スズ(\ajRoman{4})水溶液と硫化水素\\
 \hce{SnCl4 + 2H2S -> SnS + S + 4 HCl}
 \subsection{酸化鉛(\UTF{2163})}
 \subsubsection{性質}
 \hl{還元}剤として働く\\
 \hce{Sn^2+ -> Sn^4+ + 2e-}
 \subsubsection{製法}
 酢酸鉛(\ajRoman{2})水溶液にさらし粉を加える
 \subsubsection{反応}
 酸化鉛(\ajRoman{4})に濃塩酸を加えて加熱\\
 \hce{PbO2 + 4HCl -> PbCl2 + 2H2O + Cl2 ^}
 \subsection{鉛の難溶性化合物}
 \subsubsection{性質}
 \begin{itemize}
  \item 加熱すると溶けやすい
  \item \hl{酢酸鉛(\ajRoman{2})}紙を用いた\hl{硫化水素}の検出(\hl{黒}色)
 \end{itemize}