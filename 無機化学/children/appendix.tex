%APPENDIX
\part{APPENDIX}
\section{気体の乾燥剤}
固体の乾燥剤は\hl{U字管}につめて、液体の乾燥剤は\hl{洗気瓶}に入れて使用。\\
\begin{table}[h]
  \centering
\begin{tabular}{|c|cc|c|c|}\hline
  性質                   & 乾燥剤          & 化学式                     & 対象                      & 対象外(不適)                                                                             \\ \hline
  \multirow{2}{*}{酸性}  & \hl{十酸化四リン}  & \hl{\ce{P4O10}}         & \multirow{2}{*}{酸性・中性}  & 塩基性の気体(\hl{\ce{NH3}})                                                               \\ \cline{2-3} \cline{5-5}
                       & \hl{濃硫酸}     & \hl{\ce{H2SO4}}         &                         & +\hl{\ce{H2S}}(\hl{還元剤})                                                            \\ \hline
  \multirow{2}{*}{中性}  & \hl{塩化カルシウム} & \hl{\ce{CaCl2}}         & \multirow{2}{*}{ほとんど全て} & \hl{\ce{NH3}}                                                                       \\ \cline{2-3} \cline{5-5}
                       & \hl{シリカゲル}   & \hl{\ce{SiO2*$n$H2O}}   &                         & 特になし                                                                                \\ \hline
  \multirow{2}{*}{塩基性} & \hl{酸化カルシウム} & \hl{\ce{CaO}}           & \multirow{2}{*}{中性・塩基性} & 酸性の気体                                                                               \\ \cline{2-3}
                       & \hl{ソーダ石灰}   & \hl{\ce{CaO}と\ce{NaOH}} &                         & \hl{\ce{Cl2}},\hl{\ce{HCl}},\hl{\ce{H2S}},\hl{\ce{SO2}},\hl{\ce{CO2}},\hl{\ce{NO2}} \\ \hline
\end{tabular}
\end{table}
\section{水の硬度}
水の中の重荷\ce{Ca^2+}と\ce{Mg^2+}を\ce{CaCO3}として換算した時の濃度[mg/L]\\
硬水に含まれる陰イオンが
$\left\{
  \begin{tabular}{l}
    煮沸する\hl{炭酸塩}が沈澱して軟化可能(一時硬水) \\
    $\left(
      \begin{tabular}{l}
          \R 炭酸水素カルシウム水溶液                        \\
          \hce{Ca(HCO3)2 -> CaCO3 v + H2O + CO2} \\
          \R 炭酸水素マグネシウム水溶液                       \\
          \hce{Mg(HCO3)2 -> MgCO3 v + H2O + CO2}
        \end{tabular}
    \right)$                    \\
    煮沸しても軟化不可能(永久硬水)
  \end{tabular}
  \right.$
\newpage
\section{金属イオンの難容性化合物}
\begin{center}
  \begin{tabular}{|c||c|c|c|c|c|c|c|}\hline
                & \ce{Cl-}         & \ce{SO4^2-}     & \ce{H2S}        & \ce{H2S}          & \ce{OH-}          & \ce{OH-}               & \ce{NH3}                \\ \hline
                &                  &                 & 酸性              & 中・塩基性             & NH3               & 過剰                     & 過剰                      \\ \hline \hline
    \ce{K+}     & \hl{---}         & \hl{---}        & \hl{---}        & \hl{---}          & \hl{---}          & \hl{---}               & \hl{---}                \\
                & \hl{---}色        & \hl{---}色       & \hl{---}色       & \hl{---}色         & \hl{---}色         & \hl{---}色              & \hl{---}色               \\ \hline
    \ce{Ba^2+}  & \hl{---}         & \hl{\ce{BaSO4}} & \hl{---}        & \hl{---}          & \hl{---}          & \hl{---}               & \hl{---}                \\
                & \hl{---}色        & \hl{白}色         & \hl{---}色       & \hl{---}色         & \hl{---}色         & \hl{---}色              & \hl{---}色               \\ \hline
    \ce{Sr^2+}  & \hl{---}         & \hl{\ce{SrSO4}} & \hl{---}        & \hl{---}          & \hl{---}          & \hl{---}               & \hl{---}                \\
                & \hl{---}色        & \hl{白}色         & \hl{---}色       & \hl{---}色         & \hl{---}色         & \hl{---}色              & \hl{---}色               \\ \hline
    \ce{Ca^2+}  & \hl{---}         & \hl{\ce{CaSO4}} & \hl{---}        & \hl{---}          & \hl{\ce{Ca(OH)2}} & \hl{\ce{Ca(OH)2}}      & \hl{\ce{Ca(OH)2}}       \\
                & \hl{---}色        & \hl{白}色         & \hl{---}色       & \hl{---}色         & \hl{白}色           & \hl{白}色                & \hl{白}色                 \\ \hline
    \ce{Na+}    & \hl{---}         & \hl{---}        & \hl{---}        & \hl{---}          & \hl{---}          & \hl{---}               & \hl{---}                \\
                & \hl{---}色        & \hl{---}色       & \hl{---}色       & \hl{---}色         & \hl{---}色         & \hl{---}色              & \hl{---}色               \\ \hline
    \ce{Mg^2+}  & \hl{---}         & \hl{---}        & \hl{---}        & \hl{---}          & \hl{\ce{Mg(OH)2}} & \hl{\ce{Mg(OH)2}}      & \hl{---}                \\
                & \hl{---}色        & \hl{---}色       & \hl{---}色       & \hl{---}色         & \hl{白}色           & \hl{白}色                & \hl{---}色               \\ \hline
    \ce{Al^3+}  & \hl{---}         & \hl{---}        & \hl{---}        & \hl{\ce{Al(OH)3}} & \hl{\ce{Al(OH)3}} & \hl{\ce{[Al(OH)4]-}}   & \hl{\ce{Al(OH)3}}       \\
                & \hl{---}色        & \hl{---}色       & \hl{---}色       & \hl{白}色           & \hl{白}色           & \hl{白}色                & \hl{白}色                 \\ \hline
    \ce{Mn^2+}  & \hl{---}         & \hl{---}        & \hl{---}        & \hl{\ce{MnS}}     & \hl{\ce{Mn(OH)2}} & \hl{\ce{Mn(OH)2}}      & \hl{\ce{Mn(OH)2}}       \\
                & \hl{---}色        & \hl{---}色       & \hl{---}色       & \hl{淡桃}色          & \hl{白}色           & \hl{白}色                & \hl{白}色                 \\ \hline
    \ce{Zn^2+}  & \hl{---}         & \hl{---}        & \hl{---}        & \hl{\ce{ZnS}}     & \hl{\ce{Zn(OH)2}} & \hl{\ce{[Zn(OH)4]^2-}} & \hl{\ce{[Zn(NH3)4]^2+}} \\
                & \hl{---}色        & \hl{---}色       & \hl{---}色       & \hl{白}色           & \hl{白}色           & \hl{無}色                & \hl{無}色                 \\ \hline
    \ce{Cr^3+}  & \hl{---}         & \hl{---}        & \hl{---}        & \hl{---}          & \hl{\ce{Cr(OH)3}} & \hl{\ce{[Cr(OH)4]-}}   & \hl{\ce{Cr(OH)3}}       \\
                & \hl{---}色        & \hl{---}色       & \hl{---}色       & \hl{---}色         & \hl{灰緑}色          & \hl{緑}色                & \hl{灰緑}色                \\ \hline
    \ce{Fe^2+}  & \hl{---}         & \hl{---}        & \hl{---}        & \hl{\ce{FeS}}     & \hl{\ce{Fe(OH)2}} & \hl{\ce{Fe(OH)2}}      & \hl{\ce{Fe(OH)2}}       \\
                & \hl{---}色        & \hl{---}色       & \hl{---}色       & \hl{黒}色           & \hl{緑白}色          & \hl{緑白}色               & \hl{緑白}色                \\ \hline
    \ce{Fe^3+}  & \hl{---}         & \hl{---}        & \hl{\ce{Fe^2+}} & \hl{\ce{FeS}}     & \hl{\ce{Fe(OH)3}} & \hl{\ce{Fe(OH)3}}      & \hl{\ce{Fe(OH)3}}       \\
                & \hl{---}色        & \hl{---}色       & \hl{淡緑}色        & \hl{黒}色           & \hl{赤褐}色          & \hl{赤褐}色               & \hl{赤褐}色                \\ \hline
    \ce{Cd^2+}  & \hl{---}         & \hl{---}        & \hl{\ce{CdS}}   & \hl{\ce{CdS}}     & \hl{\ce{Cd(OH)2}} & \hl{\ce{Cd(OH)2}}      & \hl{\ce{[Cd(NH3)4]^2-}} \\
                & \hl{---}色        & \hl{---}色       & \hl{黄}色         & \hl{黄}色           & \hl{白}色           & \hl{白}色                & \hl{無}色                 \\ \hline
    \ce{Co^2+}  & \hl{---}         & \hl{---}        & \hl{\ce{CoS}}   & \hl{\ce{Co(OH)2}} & \hl{\ce{Co(OH)2}} & \hl{\ce{Co(OH)2}}      & \hl{\ce{Co(OH)2}}       \\
                & \hl{---}色        & \hl{---}色       & \hl{黒}色         & \hl{青}色           & \hl{青}色           & \hl{青}色                & \hl{青}色                 \\ \hline
    \ce{Ni^2+}  & \hl{---}         & \hl{---}        & \hl{\ce{NiS}}   & \hl{\ce{Ni(OH)2}} & \hl{\ce{Ni(OH)2}} & \hl{\ce{Ni(OH)2}}      & \hl{\ce{[Ni(NH3)6]^2+}} \\
                & \hl{---}色        & \hl{---}色       & \hl{黒}色         & \hl{緑白}色          & \hl{緑白}色          & \hl{緑白}色               & \hl{青紫}色                \\ \hline
    \ce{Sn^2+}  & \hl{---}         & \hl{---}        & \hl{\ce{SnS}}   & \hl{\ce{SnS}}     & \hl{\ce{Sn(OH)2}} & \hl{\ce{[Sn(OH)4]^2-}} & \hl{\ce{Sn(OH)2}}       \\
                & \hl{---}色        & \hl{---}色       & \hl{褐}色         & \hl{褐}色           & \hl{白}色           & \hl{白}色                & \hl{白}色                 \\ \hline
    \ce{Pb^2+}  & \hl{\ce{PbCl}}   & \hl{\ce{PbSO4}} & \hl{\ce{PbS}}   & \hl{\ce{PbS}}     & \hl{\ce{Pb(OH)2}} & \hl{\ce{[Pb(OH)4]^2-}} & \hl{\ce{Pb(OH)2}}       \\
                & \hl{白}色          & \hl{白}色         & \hl{黒}色         & \hl{黒}色           & \hl{白}色           & \hl{無}色                & \hl{白}色                 \\ \hline
    \ce{Cu^2+}  & \hl{---}         & \hl{---}        & \hl{\ce{CuS}}   & \hl{\ce{CuS}}     & \hl{\ce{Cu(OH)2}} & \hl{\ce{Cu(OH)2}}      & \hl{\ce{[Cu(NH3)4]^2+}} \\
                & \hl{---}色        & \hl{---}色       & \hl{白}色         & \hl{白}色           & \hl{青白}色          & \hl{青白}色               & \hl{深青}色                \\ \hline
    \ce{Hg^2+}  & \hl{---}         & \hl{---}        & \hl{\ce{HgS}}   & \hl{\ce{HgS}}     & \hl{\ce{HgO}}     & \hl{\ce{HgO}}          & \hl{\ce{HgO}}           \\
                & \hl{---}色        & \hl{---}色       & \hl{黒}色         & \hl{黒}色           & \hl{黄}色           & \hl{黄}色                & \hl{黄}色                 \\ \hline
    \ce{Hg2^2+} & \hl{\ce{Hg2Cl2}} & \hl{---}        & \hl{\ce{HgS}}   & \hl{\ce{HgS}}     & \hl{\ce{HgO}}     & \hl{\ce{HgO}}          & \hl{\ce{HgO}}           \\
                & \hl{白}色          & \hl{---}色       & \hl{黒}色         & \hl{黒}色           & \hl{黄}色           & \hl{黄}色                & \hl{黄}色                 \\ \hline
    \ce{Ag+}    & \hl{\ce{AgCl}}   & \hl{---}        & \hl{\ce{Ag2S}}  & \hl{\ce{Ag2S}}    & \hl{\ce{Ag2O}}    & \hl{\ce{Ag2O}}         & \hl{\ce{[Ag(NH3)2]+}}   \\
                & \hl{白}色          & \hl{---}色       & \hl{黒}色         & \hl{黒}色           & \hl{褐}色           & \hl{褐}色                & \hl{無}色                 \\ \hline
  \end{tabular}
\end{center}
\section{錯イオンの命名法}
(主に遷移)金属イオンに対して、\hl{非共有電子対}を持つ\hl{分子}や\hl{イオン}が\hl{配位}結合\\
「 \stamp{OrangeRed}{配位子の数(数詞)} \stamp{YellowOrange}{配位子} \stamp{YellowGreen}{金属(価数)} \stamp{Cerulean}{酸(陰イオンの場合)}イオン」
\begin{table}[h]
  \centering
  \begin{tabular}{|c|cc|cc|cccccc|}\hline
    金属イオン                & \ce{Ag+}                     & \ce{Cu+}                     & \ce{Cu^2+}                     & \ce{Zn^2+}                     & \ce{Fe^2+} & \ce{Fe^3+} & \ce{Co^3+} & \ce{Ni^2+} & \ce{Cr^3+} & \ce{Al^3+} \\ \hline
    配位数                  & \multicolumn{2}{|c|}{\hl{2}} & \multicolumn{2}{|c|}{\hl{4}} & \multicolumn{6}{|c|}{\hl{6}}                                                                                                                  \\ \hline
    \multicolumn{1}{c}{} & \multicolumn{2}{c}{\hl{直線}系} & \hl{正方}形                     & \multicolumn{1}{c}{\hl{正四面体}形} & \multicolumn{6}{c}{\hl{正八面体}形}
  \end{tabular}\\\vspace{1\zw}
  \begin{tabular}{|c|c|c|c|c|c|c|c|c|}\hline
    数  & 1       & 2       & 3        & 4        & 5        & 6        & 7        & 8        \\ \hline
    数詞 & \hl{モノ} & \hl{ジ}  & \hl{トリ}  & \hl{テトラ} & \hl{ペンタ} & \hl{ヘキサ} & \hl{ヘプタ} & \hl{オクタ} \\
      &         & \hl{ビス} & \hl{トリス} &          &          &          &          &          \\ \hline
  \end{tabular}\\\vspace{1\zw}
  \begin{tabular}{|c|c|c|c|c|c|c|}\hline
    配位子 & \ce{NH3}  & \ce{CN-}  & \ce{H2O} & \ce{OH-}    & \ce{Cl-}  & \ce{H2N - CH2CH2 - NH2} \\ \hline
    名称  & \hl{アンミン} & \hl{シアニド} & \hl{アクア} & \hl{ヒドロキシド} & \hl{クロリド} & \hl{エチレンジアミン}           \\ \hline
  \end{tabular}\\\vspace{1\zw}
\end{table}\\
エチレンジアミン$\cdots$1分子あたり2か所で\hl{配位}結合する(2座配位子)(\hl{キレート}錯体)
\begin{itemize}
  \item \ce{[Zn(OH)4]^2-}\\
        \hl{テトラヒドロキシド亜鉛(\ajRoman{2})酸イオン}
  \item \ce{[Zn(NH3)4]^2+}\\
        \hl{テトラアンミン亜鉛(\ajRoman{2})イオン}
  \item \ce{[Ag(S2O3)2]^3-}\\
        \hl{ビス(チオスルファト)銀(\ajRoman{1})イオン}
  \item \ce{[Cu(H2NCH2CH2NH2)]^2+}\\
        \hl{ビス(エチレンジアミン)銅(\ajRoman{2})イオン}
\end{itemize}
\begin{quotation}
  \hlbox{\chemfig{
    Cu(-[1,1.5,,,latex-]{NH_{2}}-[:120]{CH_{2}}?[a])(-[3,1.5,,,latex-]{H_{2}N}-[:60]{H_{2}C}?[a])(-[-1,1.5,,,latex-]{NH_{2}}-[:-120]{CH_{2}}?[b])(-[-3,1.5,,,latex-]{H_{2}N}-[:-60]{H_{2}C}?[b])
    }}
\end{quotation}

\newpage
\section{金属イオンの系統分離}
\begin{tikzpicture}[process/.style={right, draw=black, rectangle, semithick, rounded corners=3pt, dash dot},arrow/.style={-Stealth,thick},extract/.style={left, draw=black, rectangle, semithick, rounded corners=3pt}]
  \node[below right, draw=black, rectangle, rounded corners=4pt, line width=1pt, inner sep=4pt, outer sep=0pt] (begin) {金属イオンの混合溶液};
  \node[below right, draw=black, rectangle, rounded corners=4pt, line width=1pt, inner sep=4pt, outer sep=0pt] (end) at (0,-22) {\hl{\ce{K^+}},\hl{\ce{Na^+}},\hl{\ce{Mg^+}}};
  \coordinate (start point) at ($(begin.south west)!0.15!(begin.south east)$);
  \coordinate (end point) at ($(start point) + (end.north west) - (begin.south west)$);
  \coordinate (extract1) at ([yshift=-2cm]start point);
  \coordinate (extract2) at ([yshift=-6.8cm]start point);
  \coordinate (extract3) at ([yshift=-13cm]start point);
  \coordinate (extract4) at ([yshift=-16cm]start point);
  \coordinate (extract5) at ([yshift=-20.5cm]start point);

  \node[process] (1st) at ([xshift=1cm]$(start point)!0.33!(extract1)$){
      \hl{希塩酸}$\Rightarrow$\hl{\ce{Cl-}}(\hl{酸}性)
    };
  \node[process] (2nd) at ([xshift=1cm]$(extract1)!0.5!(extract2)$){
      \begin{minipage}{4.5cm}
        \hl{硫化水素}\\
        $\Rightarrow$\hl{\ce{S^2-}}(\hl{酸}性)
      \end{minipage}
    };
  \node[process] (3rd) at ([xshift=1cm]$(extract2)!0.5!(extract3)$){
      \begin{minipage}{8cm}
        \begin{enumerate}
          \setlength{\leftskip}{-13pt}
          \item \hl{煮沸}する$\Rightarrow$\hl{硫化水素}を除去
          \item \hl{希硝酸}を加える\\
                \hce{Fe^2+ -> Fe^3+}
          \item 過剰の\hl{アンモニア水}を加える(+\ce{NH4Cl})\\
                少量の\hl{\ce{OH-}}(\hl{緩衝}作用で\hl{強塩基}性)\\
                \hl{アンミン錯イオン}の生成
        \end{enumerate}
      \end{minipage}
    };
  \node[process] (4th) at ([xshift=1cm]$(extract3)!0.5!(extract4)$){
      \begin{minipage}{4cm}
        \hl{硫化水素}を加える\\
        \hl{\ce{S^2-}}(\hl{塩基}性)
      \end{minipage}
    };
  \node[process] (5th) at ([xshift=1cm]$(extract4)!0.5!(extract5)$){
      \begin{minipage}{6cm}
        \begin{enumerate}
          \setlength{\leftskip}{-13pt}
          \item \hl{酢酸}を加えて\hl{煮沸}する\\
                \hl{硫化水素}を除去(\hl{弱酸遊離})
          \item \hl{炭酸アンモニウム}(+\ce{NH4Cl})\\
                少量の\hl{\ce{CO3^2-}}
        \end{enumerate}
      \end{minipage}
    };

  \draw[arrow] (start point) -- (end point);
  \draw[arrow] (1st.west) -- ($(start point)!(1st.west)!(end point)$);
  \draw[arrow] (2nd.west) -- ($(start point)!(2nd.west)!(end point)$);
  \draw[arrow] (3rd.west) -- ($(start point)!(3rd.west)!(end point)$);
  \draw[arrow] (4th.west) -- ($(start point)!(4th.west)!(end point)$);
  \draw[arrow] (5th.west) -- ($(start point)!(5th.west)!(end point)$);

  \node[extract] (group1) at ([xshift=15cm]extract1) {
    \begin{tabular}{ll|ll}
      \hl{\ce{AgCl}}  & \hl{白}色 & \hl{\ce{Hg2Cl2}} & \hl{白}色 \\
      \hl{\ce{PbCl2}} & \hl{白}色 &                  &
    \end{tabular}
  };
  \node[extract, above] (group1-result) at ([yshift=-3cm]group1.south){\ce{Hg,Hg(NH2)Cl}};
  \coordinate (group1-end point) at (group1-result.north);
  \coordinate (group1-start point) at ($(group1.south west)!(group1-end point)!(group1.south east)$);
  \draw[arrow] (group1-start point) -- (group1-end point);
  \node[process] (group1-1) at ([xshift=1cm]$(group1-start point)!0.2!(group1-end point)$){\hl{熱水}を加える};
  \node[extract, left] (group1-2) at ([xshift=-1cm]$(group1-start point)!0.4!(group1-end point)$){\hl{\ce{Pb^2+}}};
  \node[process] (group1-3) at ([xshift=1cm]$(group1-start point)!0.6!(group1-end point)$){\hl{アンモニア水}を加える};
  \node[extract, left] (group1-4) at ([xshift=-1cm]$(group1-start point)!0.8!(group1-end point)$){\hl{\ce{Ag+}}};

  \draw[arrow] (group1-1.west) -- ($(group1-start point)!(group1-1.west)!(group1-end point)$);
  \draw[arrow] ($(group1-start point)!(group1-2.east)!(group1-end point)$) -- (group1-2.east);
  \draw[arrow] (group1-3.west) -- ($(group1-start point)!(group1-3.west)!(group1-end point)$);
  \draw[arrow] ($(group1-start point)!(group1-4.east)!(group1-end point)$) -- (group1-4.east);

  \node[extract] (group2) at ([xshift=11cm]extract2) {
    \begin{tabular}{ll|ll}
      \hl{\ce{HgS}}    & \hl{黒}色  & \hl{\ce{CuS}} & \hl{黒}色 \\
      \hl{\ce{SnS}}    & \hl{褐}色  & \hl{\ce{CdS}} & \hl{黄}色 \\
      (\hl{\ce{PbCl2}} & \hl{黒}色) &               &
    \end{tabular}
  };
  \node[extract, above] (group2-result) at ([yshift=-4cm,xshift=3.5cm]group2.south east){\hl{\ce{Cu^2+}}};
  \coordinate (group2-end point) at (group2-result.north);
  \coordinate (group2-start point) at ([yshift=-0.5cm]$(group2.west)!(group2-end point)!(group2.east)$);
  \draw[arrow] (group2.east) -| (group2-end point);
  \node[process] (group2-1) at ([xshift=1cm]$(group2-start point)!0.14!(group2-end point)$){沈澱の色};
  \node[extract, left] (group2-2) at ([xshift=-1cm]$(group2-start point)!0.28!(group2-end point)$){\hl{\ce{SnS}}・\hl{\ce{CdS}}};
  \node[process] (group2-3) at ([xshift=1cm]$(group2-start point)!0.42!(group2-end point)$){希硝酸を加えて加熱};
  \node[extract, left] (group2-4) at ([xshift=-1cm]$(group2-start point)!0.56!(group2-end point)$){\ce{HgS}};
  \node[process] (group2-5) at ([xshift=1cm]$(group2-start point)!0.70!(group2-end point)$){希硫酸を加える};
  \node[extract, left] (group2-6) at ([xshift=-1cm]$(group2-start point)!0.84!(group2-end point)$){\hl{\ce{Pb^2+}}};

  \draw[arrow] (group2-1.west) -- ($(group2-start point)!(group2-1.west)!(group2-end point)$);
  \draw[arrow] ($(group2-start point)!(group2-2.east)!(group2-end point)$) -- (group2-2.east);
  \draw[arrow] (group2-3.west) -- ($(group2-start point)!(group2-3.west)!(group2-end point)$);
  \draw[arrow] ($(group2-start point)!(group2-4.east)!(group2-end point)$) -- (group2-4.east);
  \draw[arrow] (group2-5.west) -- ($(group2-start point)!(group2-5.west)!(group2-end point)$);
  \draw[arrow] ($(group2-start point)!(group2-6.east)!(group2-end point)$) -- (group2-6.east);

  \node[extract] (group3) at ([xshift=12cm]extract3) {
    \begin{tabular}{ll|ll}
      \hl{\ce{Fe(OH)3}} & \hl{淡緑}色 & \hl{\ce{Cr(OH)3}} & \hl{黒}色 \\
      \hl{\ce{Al(OH)3}} & \hl{白}色  &                   &
    \end{tabular}
  };
  \node[process] (group3-1) at ([xshift=1cm]group3.east){沈澱の色};
  \draw[arrow] (group3-1.west) -- ($(group3.north east)!(group3-1.west)!(group3.south east)$);

  \node[extract] (group4) at ([xshift=11cm]extract4) {
    \begin{tabular}{ll|ll}
      \hl{\ce{NiS}} & \hl{黒}色 & \hl{\ce{CoS}} & \hl{黒}色  \\
      \hl{\ce{ZnS}} & \hl{白}色 & \hl{\ce{MnS}} & \hl{淡黄}色 \\
    \end{tabular}
  };
  \node[process] (group4-1) at ([xshift=1cm]group4.east){
    \begin{minipage}{6.2cm}
      沈澱の色・元の混合溶液の色\\
      \begin{tabular}{l|l}
        \ce{Ni^2+}\hl{緑}色 & \ce{Co^2+}\hl{淡赤}色
      \end{tabular}
    \end{minipage}};
  \draw[arrow] (group4-1.west) -- ($(group4.north east)!(group4-1.west)!(group4.south east)$);

  \node[extract] (group5) at ([xshift=11cm]extract5) {
    \begin{tabular}{ll|ll}
      \hl{\ce{CaCO3}} & \hl{白}色 & \hl{\ce{BaCO3}} & \hl{白}色 \\
      \hl{\ce{StCO3}} & \hl{白}色 &                 &
    \end{tabular}
  };
  \node[extract, above] (group5-result) at ([yshift=3cm]$(group5.south)!1.3!(group5.south east)$){\hl{\ce{BaCrO4}}};
  \coordinate (group5-end point) at (group5-result.south);
  \coordinate (group5-start point) at ([yshift=0.5cm]$(group5.west)!(group5-end point)!(group5.east)$);
  \draw[arrow] (group5.east) -| (group5-end point);
  \node[process] (group5-1) at ([xshift=1cm]$(group5-start point)!0.33!(group5-end point)$){
      \begin{minipage}{3.6cm}
        希酢酸に溶解して、\\クロム酸カリウム水溶液を加える
      \end{minipage}};
  \node[extract, left] (group5-2) at ([xshift=-1cm]$(group5-start point)!0.67!(group5-end point)$){\hl{\ce{SnS}}・\hl{\ce{CdS}}};

  \draw[arrow] (group5-1.west) -- ($(group5-start point)!(group5-1.west)!(group5-end point)$);
  \draw[arrow] ($(group5-start point)!(group5-2.east)!(group5-end point)$) -- (group5-2.east);

  \node[extract, right] (group6-result) at ([xshift=7cm,yshift=-2cm]end.west){\ce{Na+,K+}};
  \coordinate (group6-end point) at (group6-result.north);
  \coordinate (group6-start point) at ([yshift=0.5cm]$(end.west)!(group6-end point)!(end.east)$);
  \draw[arrow] (end.east) -| (group6-end point);
  \node[process] (group6-1) at ([xshift=1cm]$(group6-start point)!0.33!(group6-end point)$){
      \begin{minipage}{7cm}
        弱酸性下で、リン酸水素にナトリウム水溶液と\\アンモニア水を加える。
      \end{minipage}};
  \node[extract, left] (group6-2) at ([xshift=-1cm]$(group6-start point)!0.67!(group6-end point)$){\ce{MgNH4PO4*6H2O}};

  \draw[arrow] (group6-1.west) -- ($(group6-start point)!(group6-1.west)!(group6-end point)$);
  \draw[arrow] ($(group6-start point)!(group6-2.east)!(group6-end point)$) -- (group6-2.east);

  \node[process] (group6-3) at ([xshift=1cm]group6-result.east){
    炎色反応
    \begin{tabular}{l|l}
      \ce{Na}\hl{黄}色 & \ce{K}\hl{赤紫}色
    \end{tabular}};
  \draw[arrow] (group6-3.west) -- (group6-result.east);

  \draw[arrow] (extract1) -- (group1.west);
  \draw[arrow] (extract2) -- (group2.west);
  \draw[arrow] (extract3) -- (group3.west);
  \draw[arrow] (extract4) -- (group4.west);
  \draw[arrow] (extract5) -- (group5.west);
\end{tikzpicture}